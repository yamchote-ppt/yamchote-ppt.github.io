\chapter{Combinations}

\begin{center}
	   \title{\sffamily\colorbox{black}{\bfseries\textcolor{white}{\Large THEORY PART}}}
\end{center}

ในบทนี้จะกล่าวถึงเทคนิคต่าง ๆ เกี่ยวกับการนับจำนวนเหตุการณ์ โดยเริ่มจากเทคนิคเบื้องต้นที่สุดซึ่งคือหลักการบวกและหลักการคูณที่เป็นพื้นฐานของสูตรการนับอื่น ๆ ที่จะกล่าวถึงต่อไปในบทนี้ กล่าวคือถึงแม้เราจะไม่รู้สูตรในการคำนวณการนับแบบยาก ๆ แต่ถ้าเราใช้ทักษะด้านการวางแผนช่วยในการนับ ทุกปัญหาจะสามารถถูกแก้ปัญหาได้โดยใช้เพียงแค่หลักการบวกและหลักการคูณได้ หลังจากทำความคุ้นเคยกับการวางแผนการนับเหตุการณ์เบื้องต้นด้วยหลักการบวกและหลักการคูณแล้ว จะเริ่มกล่าวถึงสูตรของรูปแบบการนับต่าง ๆ ที่เฉพาะเจาะจงมากขึ้น ได้แก่ การเรียงสับเปลี่ยน และการจัดกลุ่ม ทั้งในรูปแบบไม่มีของซ้ำกันและมีของซ้ำกันหรือเลือกซ้ำได้

\section{หลักการบวกและหลักการคูณ}

~\indent อย่างที่ได้กล่าวไปตอนต้นว่าทุกสูตรที่จะถูกกล่าวถึงในบทนี้นั้นมีแนวคิดตั้งต้นมาจากหลักการบวกและหลักการคูณทั้งสิ้น เพียงแต่ต้องอาศัยทักษะในการวางแผนการนับให้เป็นขั้นเป็นตอน ดังนั้นจุดประสงค์ของหัวข้อนี้คือการทำความคุ้นเคยกับการวางแผนการนับผ่านโจทย์ที่อยู่ในระดับง่ายถึงปานกลาง โดยที่เครื่องมือการนับในเวลานี้มีเพียงแค่หลักการบวกและหลักการคูณ

\subsection{หลักการบวก}
\boxrule{หลักการบวก\index{หลักการบวก}\index{additive rule}}{ในการทำงานอย่างหนึ่งมีทางเลือกการทำอยู่ 2 ทางเลือก โดยที่ทางเลือกแรกมีวิธีทำได้ $ p $ วิธีแตกต่างกัน และทางเลือกที่สองมีวิธีทำได้ $ q $ วิธีแตกต่างกัน โดยที่ทางเลือกทั้งสองไม่มีวิธีการทำร่วมกัน และเลือกทำได้แค่ทางเลือกใดทางเลือกหนึ่งเท่านั้น ถ้าต้องการเลือกวิธีการทำงานชิ้นนี้จะสามารถเลือกทำได้ $ p+q $ วิธีที่แตกต่างกัน}

สิ่งแรกที่ต้องนึกถึงเมื่อจะเลือกใช้หลักการบวกคือกระบวนการนับของเราเป็นการแยกกรณี กล่าวคือเป็นทางเลือกให้ทำเพียงอย่างใดอย่างหนึ่ง โดยที่ไม่ว่าจะเลือกทำทางไหนก็ถือว่าจบกระบวนการทำงานชิ้นนั้น และอย่างที่สองที่ต้องระวังคือทางเลือกที่แยกออกไปต้องไม่มีวิธีการที่ซ้ำกัน กล่าวคือไม่มีการนับซ้ำเกิดขึ้นในกระบวนการนับ

ในส่วนของโจทย์ด้านล่างนั้น ผู้อ่านคงทราบดีว่าเราต้องใช้หลักการบวกในการนับเพราะเป็นโจทย์ในหัวข้อหลักการบวก แต่สิ่งที่ผมอยากให้ผู้อ่านนึกหลังจากอ่านโจทย์เสร็จคืออะไรเป็นคีย์เวิร์ดสำคัญที่บอกเราว่าขั้นตอนนี้ต้องใช้หลักการบวก

\begin{exam}
	มหาวิทยาลัยแห่งหนึ่งมีนิสิตวิชาเอกคณิตศาสตร์ 33 คน และมีนิสิตวิชาเอกวิทยาการคอมพิวเตอร์ 40 คน ถ้าต้องการเลือกนักศึกษาหนึ่งคนเพื่อเป็นคณะกรรมการของสโมสรนิสิต จะมีวิธีเลือกนิสิตดังกล่าวได้แตกต่างกันกี่วิธี
\end{exam}
\solution{...}


\begin{exam} \label{addSet}
	ให้เซต $ A=\{a,b,c,d\} $ และ $ B=\{\alpha,\beta,\gamma\} $ ถ้าต้องการเลือกตัวอักษรหนึ่งตัวจากเซต $ A $ หรือเซต $ B $ จะมีวิธีเลือกได้กี่วิธี
\end{exam}
\solution{...}

นอกจากที่เรากล่าวถึงหลักการบวกในแง่เปรียบเทียบกับการเลือกวิธีการทำงานในรูปแบบภาษามนุษย์แล้วนั้น จากตัวอย่างที่ \ref{addSet} เราจะพบว่าเราสามารถนิยามหลักการบวกได้โดยใช้เซตเข้ามาช่วยในการพูดให้เป็นภาษาคณิตศาสตร์มากขึ้นได้ดังนี้\\

\boxrule{หลักการบวกแบบภาษาเซต}{กำหนดให้ $ A $ และ $ B $ เป็นเซตที่มีสมาชิกแตกต่างกัน กล่าวคือ $ A\cap B = \emptyset $ จะได้ว่า $$ \left|A\cup B\right| = \left| A \right|+\left| B \right| $$}

และนอกจากที่เรานิยามหลักการบวกโดยใช้แค่ 2 ทางเลือก เรายังสามารถขยายแนวคิดออกไปให้มีมากกว่า 2 ทางเลือกได้ในทำนองเดียวกันคือ\\

\boxrule{หลักการบวกกรณีทั่วไป}{ถ้ามีทางเลือก $ m $ ทางเลือก ซึ่งไม่มีทางเลือกใดที่มีวิธีการซ้ำกับทางเลือกอื่น ๆ สมมติว่าทางเลือกที่หนึ่งมีวิธีทำได้ $ r_1 $ วิธี ทางเลือกที่สองมีวิธีทำได้ $ r_2 $ วิธี $ \dots $ และทางเลือกที่ $ m $ มีวิธีทำได้ $ r_m $ วิธี ดังนั้นจะมีวิธีเลือกทำงานชิ้นนี้เพียงอย่างใดอย่างหนึ่งได้แตกต่างกัน $ r_1+r_2+\cdots+r_m $ วิธี\\
	
	หรือกล่าวแบบภาษาเซตคือ ถ้า $ A_1, \dots, A_m $ เป็นเซตที่ไม่มีสองเซตใด ๆ ที่มีสมาชิกร่วมกัน กล่าวคือ $ A_i\cap A_j = \emptyset $ สำหรับทุก ๆ $ i\neq j $ จะได้ว่า $$ \left| A_1 \cup \cdots \cup A_m \right| = \left| A_1 \right| + \cdots + \left| A_m \right| $$}

\begin{exam}
	จงหา $ \left| \left\{ (x,y) \in \Z\times\Z \colon x^2+y^2\leq 4 \right\} \right| $
\end{exam}
\solution{...}
\subsection{หลักการคูณ}

\boxrule{หลักการคูณ\index{หลักการคูณ}\index{multiplicative rule}}{กระบวนการทำงานอย่างหนึ่งประกอบด้วยขั้นตอนย่อยๆ สองขั้นตอน โดยขั้นตอนแรกมีวิธีทำได้แตกต่างกัน $ p $ วิธี และไม่ว่าจะเลือกวิธีใดก็ตามในขั้นตอนแรกจะสามารถทำขั้นตอนที่สองได้แตกต่างกัน $ q $ วิธี และขั้นตอนทั้งสองนี้ไม่สามารถทำงานร่วมกันได้ ดังนั้นจะมีวิธีทำงานชิ้นนี้ได้แตกต่างกัน $ pq $ วิธี}

ประเด็นสำคัญของหลักการคูณคือการที่งานชิ้นนั้นมีความเป็นขั้นตอนทำอย่างต่อเนื่องกัน และต้องทำทุกขั้นตอนถึงจะเสร็จงานชิ้นนั้น ถ้าในการวางแผนการนับมีการแบ่งการนับออกเป็นขั้นและมั่นใจว่าเมื่อทำจบทุกขั้นแล้วจะได้ผลลัพธ์ของการจัดเรียงออกมาตามที่เราต้องการก็เป็นการยืนยันได้ในระดับหนึ่งว่าเราจะต้องใช้หลักการคูณเข้ามานับ นอกจากนั้น ข้อระวังของกฏการคูณที่ต้องพึงระวังไว้เสมอคือจำนวนวิธีการเลือกทำในขั้นตอนถัดไปจะต้องเท่ากันทั้งหมดไม่ว่าจะเลือกทำวิธีการใดในขั้นตอนปัจจุบันก็ตาม กล่าวเทียบกับนิยามด้านบนคือ ไม่ว่าเราจะเลือกวิธีใดใน $ p $ วิธีของขั้นตอนที่หนึ่ง เราจะต้องสามารถทำขั้นตอนที่สองได้ $ q $ วิธีทั้งหมด\\

\boxnote{คำถาม}{จริง ๆ แล้วเราสามารถมองหลักการคูณจากมุมมองของหลักการบวกได้ ซึ่งจะพบเหตุผลว่าทำไมเงื่อนไขของการที่จำนวนวิธีที่เลือกทำได้ในขั้นตอนถัดไปต้องเท่ากันไม่ว่าเลือกทำวิธีใดมาเป็นเงื่อนไขที่สำคัญ จงพิจารณาหลักการคูณโดยใช้การอธิบายในรูปแบบของหลักการบวก}

\begin{exam}
	มหาวิทยาลัยแห่งหนึ่งมีนิสิตวิชาเอกคณิตศาสตร์ 33 คน และมีนิสิตวิชาเอกวิทยาการคอมพิวเตอร์ 40 คน ถ้าต้องการเลือกนักศึกษาสองคนจากวิชาเอกละหนึ่งคนเพื่อเป็นคณะกรรมการของสโมสรนักศึกษา จะมีวิธีเลือกนักศึกษาได้แตกต่างกันกี่วิธี
\end{exam}
\solution{...}
\begin{exam}
	ให้เซต $ A=\{a,b,c,d\} $ และ $ B=\{\alpha,\beta,\gamma\} $ ถ้าต้องการเลือกตัวอักษร 2 ตัวจากเซต $ A $ และเซต $ B $ เซตละหนึ่งตัว จะมีวิธีเลือกที่แตกต่างกันกี่วิธี
\end{exam}
\solution{...}
ในทำนองเดียวกัน หลักการคูณก็สามารถเขียนได้ในรูปแบบของเซตดังนี้\\

\boxrule{หลักการคูณแบบภาษาเซต}{กำหนดให้ $ A $ และ $ B $ เป็นเซต และ $ A\times B = \left\{ (a,b) \colon a\in A, b\in B \right\} $ แล้วจะได้ว่า $$ \left|A\times B\right| = \left| A \right|\times\left| B \right| $$}

\begin{exam}
	จำนวนเต็มคี่ที่อยู่ระหว่าง 1000 และ 10000 ซึ่งมีเลขในแต่ละหลักแตกต่างกันมีทั้งหมดกี่จำนวน
\end{exam}
\solution{...}
\boxrule{หลักการคูณกรณีทั่วไป}{ถ้างานชิ้นหนึ่งประกอบด้วย $ m $ ขั้นตอน สมมติว่าขั้นตอนที่หนึ่งมีวิธีทำได้ $ r_1 $ วิธี ขั้นตอนที่สองมีวิธีทำได้ $ r_2 $ วิธีไม่ว่าจะเลือกวิธีการใดในขั้นตอนที่หนึ่งก็ตาม $ \dots $ และขั้นตอนที่ $ m $ มีวิธีทำได้ $ r_m $ วิธีไม่ว่าจะเลือกวิธีการใดในขั้นตอนก่อนหน้าก็ตาม ดังนั้นจะมีวิธีเลือกทำงานชิ้นนี้ได้แตกต่างกัน $ r_1 \times r_2\times\cdots\times r_m $ วิธี\\
	
	หรือกล่าวแบบภาษาเซตคือ ถ้า $ A_1, \dots, A_m $ เป็นเซตใด ๆ แล้วจะได้ว่า $$ \left| A_1 \times \cdots \times A_m \right| = \left| A_1 \right| \times \cdots \times \left| A_m \right| $$}

%\subsection{แบบฝึกหัดเพิ่มเติม}
\begin{exam}
	จำนวนเต็มคู่ที่อยู่ระหว่าง 1000 และ 10000 ซึ่งมีเลขในแต่ละหลักแตกต่างกันมีทั้งหมดกี่จำนวน
\end{exam}
\solution{...}

\begin{exam}
	จงแสดงว่าเซตที่มีสมาชิก $ n $ ตัวมีเซตย่อย $ 2^n $ เซต
\end{exam}
\solution{...}

\begin{exam}
	มีคู่สามีภรรยา 15 คู่ในงานปาร์ตีแห่งหนึ่ง จงหาจำนวนวิธีการเลือกผู้หญิงหนึ่งคนและผู้ชายอีกหนึ่งคนโดยที่ (1) ต้องเป็นคู่สามีภรรยากัน (2) ต้องไม่เป็นคู่สามีภรรยากัน
\end{exam}
\solution{...}

\begin{exam}
	พาสเวิร์ดของระบบความปลอดภัยแห่งหนึ่งเป็นตัวอักษรภาษาอังกฤษยาว 3 หรือ 4 ตำแหน่ง จงหา (1) จำนวนของพาสเวิร์ดที่เป็นไปได้ทั้งหมด (2) จำนวนของพาสเวิร์ดที่เป็นไปได้ทั้งหมดที่ใช้ตัวอักษรไม่ซ้ำกัน
\end{exam}
\solution{...}

\begin{exam}
	จงหาจำนวนของตัวประกอบที่เป็นจำนวนเต็มบวกของ $441,000 ( =2^3\times 3^2 \times 5^3 \times 7^2 )$
\end{exam}
\solution{...}

%\boxrule{จำนวนของตัวประกอบที่เป็นจำนวนเต็มบวก}{~\\~\\~\\~\\}

\begin{exam}
	จงหาจำนวนวิธีในการเขียน $ 441,000 $ ในรูปผลคูณของจำนวนเต็มบวก 2 จำนวนที่เป็นจำนวนเฉพาะสัมพัทธ์กัน (เช่น $1\times 441,000$ หรือ $ 441 \times 1000 $)
\end{exam}
\solution{...}

\begin{exam}
	กำหนดให้ $ X = \{1,2,3,\dots,10\} $ และ $ S=\{(a,b,c)\colon a,b,c\in X, a<b \text{ และ } a<c\} $ จงหาจำนวนสมาชิกทั้งหมดของ $ S $
\end{exam}
\solution{...}

\section{การเรียงสับเปลี่ยน}
\subsection{การเรียงสับเปลี่ยนเชิงเส้นแบบของไม่ซ้ำ}
กำหนดให้ $ A=\{a_1,a_2,\dots , a_n\} $ เป็นเซตของ $ n $ สิ่งของที่แตกต่างกัน และให้ $ 0\leq r\leq n $ แล้ว \textbf{การเรียงสับเปลี่ยน $ r $ ชิ้น}ของเซต $ A $ ($ r $-permutation) \index{การเรียงสับเปลี่ยน}\index{เรียงสับเปลี่ยน}\index{permutation} คือรูปแบบในการจัดเรียงลำดับเป็นแถวตรงของสมาชิก $ r $ ตัวใดๆ จากเซต $ A $ และเขียนแทนจำนวนของรูปแบบดังกล่าวที่เป็นไปได้ทั้งหมดด้วย $ P(n,r) $

\begin{exam}\label{enumPerm}
	ให้ $ A=\{a,b,c,d\} $ จงเขียนรูปแบบการเรียงสับเปลี่ยนของ 3 ชิ้นจากเซต $ A $ ทั้งหมด
\end{exam}
\solution{...}
ในกรณีที่ $ n $ มีค่าน้อย ๆ ก็เป็นการง่ายที่จะไล่ทุกรูปแบบเพื่อนับ แต่ในกรณีที่ $ n $ มีค่ามาก ๆ คงไม่เป็นเรื่องง่ายที่จะเขียนไล่ให้ครบแน่ ๆ จึงต้องมาพิจารณากันว่าแล้วเราจะคำนวณหาค่า $ P(n,r) $ กันอย่างไร

อย่างที่ได้กล่าวไปหลายรอบแล้วว่าเบื้องหลังของสูตรการนับต่าง ๆ นั้นมีพื้นฐานมาจากหลักการบวกและหลักการคูณทั้งสิ้น เพียงแค่ต้องวางแผนขั้นตอนการนับให้ถูกต้อง ดังนั้น สิ่งแรกที่ต้องทำคือวางแผนว่าเราจะวางขั้นตอนของการเรียงสับเปลี่ยน $ r $ ชิ้นจากของ $ n $ ชิ้นอย่างไร

แนวคิดหนึ่งที่น่าจะเป็นแนวคิดที่ผู้อ่านทุกคนคิดถึงเป็นอย่างแรกคือ \textbf{เลือกของจากกองตัวเลือกที่มีมาใส่ทีละตำแหน่งไล่ไปตั้งแต่ตำแหน่งแรกจนถึงตำแหน่งสุดท้าย} \\

\boxrule{จำนวนวิธีในการเรียงสับเปลี่ยน}{$ P(n,r) $ คือจำนวนสมาชิกของเซต $ \left\{(x_1,x_2,\dots,x_r)| x_i\in\left\{a_1,\dots,a_n\right\} \text{ และ } x_i \neq x_j \text{ สำหรับทุกๆ } i\neq j\right\} $ และจะได้ว่า
	\[ P(n,r)=\frac{n!}{(n-r)!} \]}

\boxnote{Note}{$$ P(n,0) = 1 \text{ และ } P(n,1)=n \text{ และ } P(n,n) = n!$$}

\boxnote{คำเตือน}{การเรียงสับเปลี่ยนเป็นเพียงแค่เครื่องมือหนึ่งในการนับ ไม่ใช่รูปแบบของโจทย์ อาจมีการใช้พร้อมกับหลักการบวก และหลักการคูณ และการเรียงสับเปลี่ยนอาจเป็นเพียงการนับในขั้นตอนใดขั้นตอนหนึ่งของหลักการคูณก็ได้}

\begin{exam}
	จงหาจำนวนคำซึ่งมีความยาว 4 ตัวอักษร โดยที่ตัวอักษรทั้ง 4 ตัวมาจากเซต $ \{a,b,c,d,e\} $
\end{exam}
\solution{...}
\begin{exam}
	จัดคน 6 คนเข้านั่งเรียงในแนวเส้นตรงได้กี่วิธี
\end{exam}
\solution{...}
\begin{exam}
	จัดสามีภรรยา 3 คู่เข้านั่งเรียงแถวได้กี่วิธีถ้า (1) หัวแถวและท้ายแถวต้องเป็นผู้ชาย (2) ภรรยาต้องนั่งติดกับสามี
\end{exam}
\solution{...}
\begin{exam}
	จงหาจำนวนของจำนวนเต็มซึ่งมีความยาว 7 หลัก แต่ละหลักแตกต่างกันและไม่เป็น 0 โดยที่เลข 5 และเลข 6 ต้องไม่ปรากฏในตำแหน่งติดกัน
\end{exam}
\solution{...}
\begin{exam}\label{combiPr}
	จงอธิบายเหตุผลเชิงการจัดเรียงว่า $$ P(n,n) = P(n,k)\times P(n-k,n-k) $$
\end{exam}
\solution{...}
\boxnote{Note}{เรียกการพิสูจน์แบบตัวอย่างที่ \ref{combiPr} ว่า \textbf{combinatorial proof} หรือเรียกว่า \textbf{เทคนิค double counting}}

\begin{exam}
	จำนวนเต็มคู่ที่อยู่ระหว่าง 20000 และ 70000 ซึ่งมีเลขในแต่ละหลักแตกต่างกันทั้งหมดมีกี่จำนวน
\end{exam}
\solution{...}
\begin{exam}
	กำหนดให้ $ S $ เป็นเซตของจำนวนนับที่สร้างมาจากเลขโดด $ \{ 1,3,5,7\} $ ที่เลขในแต่ละหลักแตกต่างกันทั้งหมด จงหา
	\begin{enumerate}
		\item $ |S| $
		\item $ \sum_{n\in S} n $
	\end{enumerate}
\end{exam}
\solution{...}
\subsection{การเรียงสับเปลี่ยนแบบวงกลม}
\begin{itemize}
	\item มีข้อแตกต่างจากการเรียงสับเปลี่ยนเชิงเส้นอย่างไร (มองว่าสองรูปแบบการจัดเรียงแตกต่างกันอย่างไร)
	\item ออกแบบกระบวนการนับอย่างไร
\end{itemize}
\vspace{3 cm}
\begin{exam}
	จงเขียนรูปแบบการจัดเรียงเชิงเส้น 4 สิ่งจากเซต $ A = \{a,b,c,d\} $ ซึ่งมี $ 4!=24 $ แบบ และจงเขียนแยกว่าแบบใดบ้างที่เมื่อนำมาเรียงสับเปลี่ยนเป็นวงกลมจะได้รูปแบบเดียวกัน (และสังเกตรูปแบบเพื่อนับ)
\end{exam}
\solution{...}
\boxrule{การเรียงสับเปลี่ยนแบบวงกลม}{\textbf{การเรียงสับเปลี่ยนแบบวงกลม} คือ รูปแบบการจัดเรียงที่นำรูปแบบการจัดเรียงเชิงเส้นมาล้อมเป็นวงกลม ซึ่งจะได้ว่าสองรูปแบบการจัดเรียงเชิงเส้นที่ต่างกันที่เมื่อนำมาล้อมเป็นวงกลมแล้วจะมองว่าเป็นรูปแบบเดียวกันเกิดจาก\\
	\vspace{2 cm}~
	และจะได้ว่าจำนวนวิธีการจัดเรียงสับเปลี่ยนแบบวงกลมของสิ่งของ $ n $ สิ่งทั้งหมดเท่ากับ
	\vspace{1.5 cm}}

\begin{exam}
	นำเด็กผู้ชาย 5 คนและเด็กผู้หญิง 3 คนมานั่งล้อมโต๊ะกลม จะนั่งได้กี่วิธีถ้า
	\begin{enumerate}
		\item ไม่มีเงื่อนไขเพิ่มเติม
		\item เด็กชาย $ B_1 $ และเด็กหญิง $ G_1 $ ไม่นั่งติดกัน
		\item ไม่มีเด็กผู้หญิงสองคนใด ๆ นั่งติดกัน
	\end{enumerate}
\end{exam}
\solution{...}

\begin{exam}
	จงหาจำนวนวิธีการนั่งที่แตกต่างกันของคู่สามีภรรยา $ n $ คู่รอบโต๊ะวงกลม โดยที่
	\begin{enumerate}
		\item ผู้ชายและผู้หญิงนั่งสลับกัน
		\item คู่สามีภรรยาต้องนั่งติดกัน
	\end{enumerate}
\end{exam}
\solution{...}

\begin{exam}
	จากตัวอย่างที่ \ref{enumPerm} ที่เราได้เขียนรูปแบบการจัดเรียงเชิงเส้น 3 สิ่งจากเซต $ A = \{a,b,c,d\} $ ซึ่งมี $ P(4,3)=24 $ แบบ จงเขียนแยกว่าแบบใดบ้างที่เมื่อนำมาเรียงสับเปลี่ยนเป็นวงกลมจะได้รูปแบบเดียวกัน (และสังเกตรูปแบบเพื่อนับ)
\end{exam}
\solution{...}

\boxrule{การเรียงสับเปลี่ยนแบบวงกลมแบบทั่วไป}{ถ้ามีของ $ n $ สิ่งแตกต่างกัน จะนำมาจัดเรียงเป็นวงกลม $ r $ สิ่งได้แตกต่างกัน $ Q(n,r) $ วิธี โดยที่
	$$ Q(n,r) = \frac{P(n,r)}{r} $$}

\subsection{การเรียงสับเปลี่ยนเชิงเส้นแบบของซ้ำ}
\boxrule{การเรียงสับเปลี่ยนเชิงเส้นแบบของซ้ำ}{ถ้ามีของ $ n $ สิ่ง ซึ่งแบ่งออกเป็น $ k $ ประเภท โดยของในประเภทเดียวกันจะมองเป็นสิ่งเดียวกัน โดยที่มีของประเภทที่หนึ่งอยู่ $ n_1 $ ชิ้น ของประเภทที่สองมีอยู่ $ n_2 $ ชิ้น $ ... $ ของประเภทที่ $ k $ มีอยู่ $ n_k $ ชิ้น โดยที่ $ n_1+n_2+\cdots+n_k=n $ แล้วจะได้ว่าจำนวนวิธีการจัดเรียงสับเปลี่ยนเชิงเส้นของสิ่งของ $ n $ สิ่งนี้เท่ากับ $$ P(n;n_1,n_2,\dots,n_k) =\ \ \ \ \ \ \ \ \ \ \ \ \ \ \ \ \ \ \ \ \ \ \ \ \ \ \ \ \ \ \ \ \ \ \ \ \ \ \ \ \ $$}

\begin{exam}
	จงหาจำนวนวิธีการจัดเรียงคำว่า MISSISSIPPI ที่แตกต่างกันทั้งหมด
\end{exam}
\solution{...}



\section{การจัดกลุ่ม}
กำหนดให้ $ A=\{a_1,a_2,\dots , a_n\} $ เป็นเซตของ $ n $ สิ่งของที่แตกต่างกัน และให้ $ 0\leq r\leq n $ แล้ว \textbf{การจัดกลุ่ม $ r $ ชิ้น}ของเซต $ A $ ($ r $-combination) \index{การจัดกลุ่ม}\index{จัดกลุ่ม}\index{combination} คือรูปแบบในการจัดสมาชิก $ r $ ตัวใดๆ จากเซต $ A $ เข้ากลุ่มเดียวกัน โดยที่ในกลุ่มเราไม่สนใจลำดับของสมาชิก แต่สนใจเพียงแค่มีใครอยู่บ้าง และเขียนแทนจำนวนของรูปแบบดังกล่าวที่เป็นไปได้ทั้งหมดด้วย $ C(n,r) $ หรือ $ \binom{n}{r} $

\begin{exam}\label{enumPerm}
	ให้ $ A=\{a,b,c,d\} $ จงเขียนรูปแบบการเรียงจัดกลุ่มของ 3 ชิ้นจากเซต $ A $ ทั้งหมด
\end{exam}
\solution{...}

\boxrule{จำนวนวิธีในการจัดกลุ่ม}{$ C(n,r) $ คือจำนวนเซตย่อยที่มีสมาชิก $ r $ ตัวของเซตที่มีสมาชิก $ n $ ตัว กล่าวคือ $$ C(n,r) = \left\{\{x_1,x_2,\dots,x_r\}| x_i\in\left\{a_1,\dots,a_n\right\} \text{ และ } x_i \neq x_j \text{ สำหรับทุกๆ } i\neq j\right\} $$ และจะได้ว่า
	\[ C(n,r)= \]}

\begin{exam}
	จงหาจํานวนทั้งหมดของบิตสตริงโดยมีความยาวเท่ากับ 9 ซึ่งมีเลขโดด 1 อยู่สี่ตําแหน่ง
\end{exam}
\solution{...}

\begin{exam}
	จงหาจำนวนวิธีการจัดเรียงคำว่า MISSISSIPPI ที่แตกต่างกันทั้งหมด (โจทย์เดิม แต่ใช้เทคนิคการจัดกลุ่มมาช่วยนับ)
\end{exam}
\solution{...}

\begin{exam}
	จงหาจํานวนวิธีทั้งหมดในการจัดแบ่งนักเรียน 7 คน ออกเป็นสามกลุ่ม โดยให้มีกลุ่มละ
	สามคน 1 กลุ่ม และกลุ่มละสองคน 2 กลุ่ม
	
\end{exam}
\solution{...}

\begin{exam}
	จงหาจํานวนวิธีทั้งหมดในการที่สุขใจเชิญเพื่อนเพียง 6 คนจากเพื่อนสนิททั้งหมด 10 
	คนมารับประทานอาหารเย็นด้วยกัน ซึ่งใน 10 คนนี้มี 2 คนเป็นพี่น้องกัน ถ้าจะเชิญมาต้องเชิญทั้ง
	พี่และน้องมาด้วย
	
\end{exam}
\solution{...}

\begin{exam}
	จงใช้เหตุผลเชิงการนับเพื่อพิสูจน์ว่า
	$$ \binom{n}{r} = \binom{n}{n-r} $$
\end{exam}
\solution{...}

\begin{exam}
	จงใช้เหตุผลเชิงการนับเพื่อพิสูจน์ว่า
	$$ \binom{n}{r} = \binom{n-1}{r-1}+\binom{n-1}{r} $$
\end{exam}
\solution{...}


\section{สัมประสิทธิ์ทวินาม}

ในหัวข้อที่ผ่านมานั้น เราได้นิยามจำนวน $ \binom{n}{r} $ หรือ $ C(n,r) $ ไปแล้วด้วยปัญหาของการสร้างเซตย่อยขนาด $ r $  สมาชิกจากเซตที่มี $ n $ สมาชิก แต่ทั้งนี้ เรายังสามารถนิยามเพิ่มเติมในกรณีของ $ r<0 $ หรือกรณี $ r>n $ ได้เป็น
\[ \binom{n}{r}=\begin{cases}
	\frac{n!}{r!(n-r)!}&\text{ถ้า $ 0\leq r \leq n $}\\
	0 &\text{ถ้า $ r>n $ หรือ $ r<0 $}
\end{cases} \]
และเรายังสามารถพิสูจน์เอกลักษณ์ต่างๆ ของค่าเชิงการจัดกลุ่มได้โดยใช้หลักการนับเข้ามาช่วย

แต่ว่าเรายังสามารถนิยามค่าของสัญลักษณ์ $ \binom{n}{r} $ ได้ในอีกรูปแบบหนึ่งผ่านการพิจารณารูปแบบการกระจายของพหหุนามทวินาม $ (x+y)^n $ โดยเราจะพบว่าค่าเชิงการจัดกลุ่ม $ \binom{n}{r} $ นั้นจะเป็นส่วนของค่าสัมประสิทธ์ของพหุนามที่ได้มาจากการกระจายพหุนามทวินามดังกล่าว ทำให้บ่อยครั้งสัญลักษณ์เชิงการจัดกลุ่มดังกล่าวอาจจะถูกเรียกว่า \textbf{สัมประสิทธิ์ทวินาม} \index{สัมประสิทธิ์ทวินาม} (binomial coefficient \index{binomial coefficient})

\subsection{ทฤษฎีบททวินาม}
\boxrule{ทฤษฎีบททวินาม}{สำหรับจำนวนเต็มบวก $ n $ ใดๆ จะได้ว่า \[ (x+y)^n=\binom{n}{0}x^n+\binom{n}{1}x^{n-1}y+\cdots+\binom{n}{n-1}xy^{n-1}+\binom{n}{n}y^n \]}
พิสูจน์โดยใช้หลักการนับ!

\begin{exam}(easy exercise)
	\begin{enumerate}
		\item จงหาสัมประสิทธิ์ของ $ x^2y^6 $ ที่ได้จากการกระจาย $ (2x+y^2)^5 $
		\item จงใช้ทฤษฎีบททวินามหา $ \binom{n}{0}+\binom{n}{1}+\cdots+\binom{n}{n} $
	\end{enumerate}
\end{exam}
\solution{...}
\subsection{การใช้ทฤษฎีบททวินามในการพิสูจน์เอกลักษณ์เชิงการจัด}
\begin{exam}
	จงแสดงว่า\begin{enumerate}
		\item $ \sum_{r=0}^n (-1)^r\binom{n}{r}=0 $
		\item $ \binom{n}{0}+\binom{n}{2}\cdots+\binom{n}{2k}+\cdots = \binom{n}{1}+\binom{n}{3}\cdots+\binom{n}{2k+1}+\cdots = 2^{n-1} $
		\item $ \sum_{r=1}^n r\binom{n}{r}=n\cdot2^{n-1} $
		\item ***$ \sum_{i=0}^r\binom{m}{i}\binom{n}{r-i}=\binom{m+n}{r} $
	\end{enumerate}
\end{exam}
\solution{...}

\subsection{โจทย์ปัญหาเพิ่มเติมเกี่ยวกับการจัดกลุ่ม}
\begin{exam}
	\begin{enumerate}
		\item \label{route} มีกี่วิธีในการเดินตามจุดพิกัดจำนวนเต็มจากจุด $ (0,0) $ ไปจุด $ (11,5) $ ใดๆ โดยที่เดินได้แค่ทิศขึ้นและทางขวาเท่านั้น
		\item จากโจทย์ข้อที่ \ref{route} ถ้าเพิ่มเงื่อนไขว่าต้องผ่านจุด $ (4,3) $ ก่อน จะเดินได้กี่วิธี
		\item จากโจทย์ข้อที่ \ref{route} ถ้าเพิ่มเงื่อนไขว่าต้องผ่านเส้นที่เชื่อมระหว่างจุด $ (2,3) $ และ $ (3,3) $ ก่อน จะเดินได้กี่วิธี
	\end{enumerate}
\end{exam}
\solution{...}

\section{หลักการนำเข้า-ตัดออก}

\section{กฏรังนกพิราบ}