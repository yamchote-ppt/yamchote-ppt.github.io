\chapter{Set Theory and Its Family}

ในบทที่ \ref{chap:mathLang} เราได้เกริ่นถึงบทบาทของสิ่งต่าง ๆ ในคณิตศาสตร์ดีสครีตเพื่อที่จะใช้ในการอธิบายสรรพสิ่งต่าง ๆ ให้อยู่ในรูปแบบทางคณิตศาสตร์ที่รัดกุมเพื่อนำไปสู่การให้เหตุผล เช่นเราใช้เซตในการอธิบายสถานภาพหรือการเป็นสมาชิกของสิ่งต่าง ๆ และเราอธิบายหลักการคิดเชิงความจริงหรือเท็จ รวมถึงวิธีการแปลภาษาด้วยตรรกศาสตร์ เราสามารถพูดถึงการใช้สมาชิกต่าง ๆ มาคำนวณหรือสร้างเป็นสมาชิกตัวอื่นโดยใช้ฟังก์ชัน และสามารถพูดถึงการเชื่อมโยงกันด้วยสิ่งที่เรียกว่าความสัมพันธ์

ทั้งนี้ ในบทดังกล่าวจะยังไม่ได้พูดถึงรายละเอียดเชิงเทคนิค(ทางคณิตศาสตร์)ของสิ่งต่าง ๆ ไม่ว่าจะเป็นนิยาม หรือการพิสูจน์คุณสมบัติต่าง ๆ ซึ่งเราจะมากล่าวถึงกันในบทนี้ โดยเราจะเริ่มจากเซต ซึ่งแท้ที่จริงแล้วสิ่งต่าง ๆ ในคณิตศาสตร์ก็ถูกสร้างขึ้นมาจากเซตทั้งสิ้น จึงมีศาสตร์เฉพาะทางที่ศึกษาเฉพาะการใช้เซตเพื่ออธิบายคณิตศาสตร์ เรียกว่า \textbf{ทฤษฎีเซต (set theory)} รวมไปถึงนิยามความสัมพันธ์และฟังก์ชันตามมา

\section{เซต}
\subsection{การเป็นสมาชิก}
\subsection{เซตย่อยและเซตกำลัง}
\subsection{การดำเนินการของเซต}

\section{ความสัมพันธ์}
\subsection{คู่อันดับ ผลคูณคาร์ทีเซียน และความสัมพันธ์}
\subsection{ความสัมพันธ์ประเภทต่าง ๆ}
\subsection{ความสัมพันธ์สมมูล และชั้นสมมูล}

\section{ฟังก์ชัน}
\subsection{ฟังก์ชัน โดเมน และเรนจ์}
\subsection{ประเภทของฟังก์ชัน}
\subsection{ฟังก์ชันประกอบ}

\section{ทฤษฎีเซตเชิงการนับ}
\subsection{การสมมูลกันเชิงการนับของเซต และคาร์ดินอลของเซต}
\subsection{Cantor's Theorem}