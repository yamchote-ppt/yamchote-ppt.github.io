\chapter{Set Theory and Its Family}

ในบทที่ \ref{chap:mathLang} เราได้เกริ่นถึงบทบาทของสิ่งต่าง ๆ ในคณิตศาสตร์ดีสครีตเพื่อที่จะใช้ในการอธิบายสรรพสิ่งต่าง ๆ ให้อยู่ในรูปแบบทางคณิตศาสตร์ที่รัดกุมเพื่อนำไปสู่การให้เหตุผล เช่นเราใช้เซตในการอธิบายสถานภาพหรือการเป็นสมาชิกของสิ่งต่าง ๆ และเราอธิบายหลักการคิดเชิงความจริงหรือเท็จ รวมถึงวิธีการแปลภาษาด้วยตรรกศาสตร์ เราสามารถพูดถึงการใช้สมาชิกต่าง ๆ มาคำนวณหรือสร้างเป็นสมาชิกตัวอื่นโดยใช้ฟังก์ชัน และสามารถพูดถึงการเชื่อมโยงกันด้วยสิ่งที่เรียกว่าความสัมพันธ์

ทั้งนี้ ในบทดังกล่าวจะยังไม่ได้พูดถึงรายละเอียดเชิงเทคนิค(ทางคณิตศาสตร์)ของสิ่งต่าง ๆ ไม่ว่าจะเป็นนิยาม หรือการพิสูจน์คุณสมบัติต่าง ๆ ซึ่งเราจะมากล่าวถึงกันในบทนี้ โดยเราจะเริ่มจากเซต ซึ่งแท้ที่จริงแล้วสิ่งต่าง ๆ ในคณิตศาสตร์ก็ถูกสร้างขึ้นมาจากเซตทั้งสิ้น จึงมีศาสตร์เฉพาะทางที่ศึกษาเฉพาะการใช้เซตเพื่ออธิบายคณิตศาสตร์ เรียกว่า \textbf{ทฤษฎีเซต (set theory)} รวมไปถึงนิยามความสัมพันธ์และฟังก์ชันตามมา

จริง ๆ แล้ว เนื้อหาในบทนี้อาจไม่ใช่ส่วนที่ผู้เรียนจะเห็นประโยชน์เชิงประยุกต์ได้โดยตรงเหมือนบทอื่น ๆ ที่จะได้เรียนต่อไป เช่น ทฤษฎีจำนวน ทฤษฎีกราฟ หรือการเวียนเกิด ซึ่งสามารถเชื่อมโยงกับปัญหาในวิทยาการคอมพิวเตอร์ได้อย่างชัดเจน อย่างไรก็ตาม ความเข้าใจในเซตและโครงสร้างของมันเป็นสิ่งที่เปรียบเสมือนรากฐานของคณิตศาสตร์ทั้งหมด เพราะแนวคิดเรื่อง “การเป็นสมาชิก” และ “การสร้างวัตถุใหม่จากวัตถุที่มีอยู่เดิม” ปรากฏอยู่ในทุกแขนงของคณิตศาสตร์และคอมพิวเตอร์

ดังนั้น บทนี้จึงควรถูกมองว่าเป็นการฝึกสร้างกรอบความคิดเชิงโครงสร้าง (structural thinking) มากกว่าการเรียนรู้เพื่อนำไปคำนวณโดยตรง อีกทั้งยังเป็นบทที่เหมาะอย่างยิ่งสำหรับการฝึกฝนทักษะการให้เหตุผลและการเขียนพิสูจน์ทางคณิตศาสตร์ ซึ่งเป็นหัวใจสำคัญของการเรียนคณิตศาสตร์ดีสครีต และจะกลายเป็นเครื่องมือพื้นฐานที่จำเป็นสำหรับบทต่อ ๆ ไปของรายวิชานี้

\section{เซต}
ในเบื้องต้น เราจะมองว่าเซตคือสิ่งที่เราเอาไว้ใช้อธิบายกลุ่มของสิ่งของเพื่อที่จะบอกว่าใครอยู่หรือไม่อยู่ในกลุ่มดังกล่าว โดยจะกล่าวว่าการอยู่ในกลุ่มคือการเป็นสมาชิกของเซต เขียนแทนด้วยสัญลักษณ์ $a \in A$ แทนการบอกว่า $a$ เป็นสมาชิกของเซต $A$ และเราสามารถบรรยายเซตของสิ่งที่สนใจได้ในลักษณะ
$$
\{ x \in \mathcal{U} \mid \text{คุณลักษณะเกี่ยวกับ $x$}\}
$$
เพื่อระบุว่าเราสนใจสมาชิก $x$ จากเซตที่ตั้งต้นไว้ $\mathcal{U}$ (ซึ่งเรียกว่าเอกภพสัมพัทธ์) และสอดคล้องเงื่อนไขเกี่ยวกับคุณสมบัติที่ระบุไว้ โดยที่เราอาจจะไม่ระบุเอกภพสัมพัทธ์ก็ได้ในกรณีที่ทราบข้อตกลงกันไว้เรียบร้อยในบริบทที่กำลังกล่าวถึงแล้ว และเซตสองเซตเท่ากัน $A=B$ ก็ต่อเมื่อสมาชิกเหมือนกันทุกตัว กล่าวคือ
$$
A = B \iff \underline{\hspace{7cm}}
$$

อย่างไรก็ตาม ในทางทฤษฎีเซต (set theory) เราไม่ได้เริ่มต้นด้วย “นิยามของคำว่าเซต” อย่างที่ทำกับคำอื่น ๆ ในคณิตศาสตร์ เหตุผลก็เพราะว่า หากเราพยายามนิยามว่า “เซตคือสิ่งที่มีคุณสมบัติแบบใดแบบหนึ่ง” เราจะตกอยู่ในข้อขัดแย้งในตัวเองได้ เช่น ข้อขัดแย้งของรัสเซล (Russell's paradox) ดังนั้น แทนที่จะนิยามว่าอะไรคือเซต เราจึงเริ่มต้นด้วยการยอมรับสัจพจน์ (axioms) ชุดหนึ่งที่บอกเพียงว่า “อะไรบ้างถือเป็นเซตเบื้องต้น” และ “เราสามารถสร้างเซตใหม่จากเซตที่มีอยู่ได้อย่างไร” เช่น สัจพจน์ของเซอร์เมโล–แฟรงเคิล (Zermelo–Fraenkel Axioms, ZF) ที่เป็นพื้นฐานของคณิตศาสตร์ส่วนใหญ่ในปัจจุบัน เราจึงพูดได้เพียงว่าเซตคือสิ่งที่ทำงานอยู่ภายใต้กฎเหล่านี้ ไม่ใช่สิ่งที่เรานิยามขึ้นโดยตรง แต่เป็นสิ่งที่เรายอมรับให้ “มีอยู่” ตามสัจพจน์ที่ใช้สร้างจักรวาลทางคณิตศาสตร์ของเรา

ในเมื่อได้กล่าวถึง Russell's paradox มาแล้ว ก็อยากขอพามาช่วยกันขบคิดซักหน่อยว่าจะเกิดอะไรขึ้นถ้าเรายอมรับว่าอะไรก็ตามที่เป็นกลุ่มของสิ่งของถือว่าเป็นเซตเสมอ

\begin{prop}
	{Russell's paradox}{}
	ระบบเซตที่ให้นิยามของเซตว่า ``\textit{เซตคือกลุ่มของสิ่งของใด ๆ}'' (ซึ่งระบบดังกล่าวเรียกว่าระบบ naive set theory) เกิดข้อขัดแย้ง
\end{prop}
ก่อนจะไปที่บทพิสูจน์ของคุณสมบัติดังกล่าวจะพามาคิดผ่านคำถามดังต่อไปนี้
\begin{enumerate}
	\item ลองพิจารณาประโยค ``ร้านที่โกนหนวดให้เฉพาะคนที่ไม่โกนหนวดตัวเอง'' ใครจะโกนหนวดให้ช่างโกนหนวด?
	\item ถ้าสมมติให้มองว่าเอกภพสัมพัทธ์ที่เรากำลังสนใจคือเซตของคนทั้งหมดในหมู่บ้าน และสมมติว่ากำลังพิจารณาช่างโกนหนวดคนหนึ่ง จงเขียนอธิบายเซตของคนที่ช่างโกนหนวดโกนหนวดให้ (ซึ่งเป็นเซตภายใต้ระบบ naive set theory) ในลักษณะภาษาคณิตศาสตร์\label{russelSet}
	\item เซตที่ได้จากข้อที่ผ่านมานี้เป็นสมาชิกของตัวเองหรือไม่?
	\item จะได้ในท้ายที่สุดว่าเซตดังกล่าวทำให้เกิดความขัดแย้ง เราจะยังสามารถนิยาม “เซตของทุกสิ่งที่มีคุณสมบัติบางอย่างร่วมกัน” ได้หรือไม่? ถ้าไม่ เราควรมีหลักเกณฑ์แบบใดมาควบคุมการสร้างเซตแทน?
\end{enumerate}

\begin{proof}[Proof of Russell's Paradox]
	~\\Hint: พิจารณาเซตที่นิยามได้จากคำถามชวนคิดข้อ \ref{russelSet}
	~\\
	~\\
	~\\
	~\\
	~\\
	~\\
	~\\
	~\\
\end{proof}
\newpage
หลังจากเราพบว่าการนิยามเซตอย่างอิสระอาจนำไปสู่ความขัดแย้ง เช่น ในกรณีของข้อขัดแย้งของรัสเซล คำถามสำคัญจึงเกิดขึ้นว่า “แล้วเราจะนิยามเซตอย่างปลอดภัยได้อย่างไร?” นักคณิตศาสตร์จึงหันกลับมาทบทวนวิธีคิดพื้นฐานของตนเองใหม่ทั้งหมด แทนที่จะเริ่มจากแนวคิดว่า “เซตคือสิ่งที่มีคุณสมบัติบางอย่างร่วมกัน” ซึ่งเปิดช่องให้เกิดความย้อนแย้ง พวกเขาเลือกที่จะเริ่มจาก “สิ่งที่ยอมรับว่าเป็นจริงโดยไม่ต้องพิสูจน์” หรือที่เรียกว่า \textit{สัจพจน์} (axioms)

แนวทางนี้นำไปสู่การสร้างระบบที่เรียกว่า \textit{ทฤษฎีเซตของเซอร์เมโล–แฟรงเคิล} (Zermelo–Fraenkel Set Theory) ซึ่งเป็นพื้นฐานของคณิตศาสตร์ส่วนใหญ่ในปัจจุบัน แทนที่จะบอกว่าอะไรคือเซต ระบบนี้กำหนดกฎเกณฑ์ว่าหากเรามีเซตอยู่บางเซตแล้ว เราสามารถสร้างเซตใหม่ได้อย่างไรบ้าง เช่น สัจพจน์ของการเป็นสมาชิกของเซตว่าง สัจพจน์ของเซตเพาเวอร์ (power set) หรือสัจพจน์ของการรวม (union) เป็นต้น ภายใต้ระบบนี้ เราจะไม่สามารถนิยาม “เซตของทุกเซต” ได้อีกต่อไป เพราะนั่นจะละเมิดขอบเขตของสัจพจน์และนำไปสู่ความขัดแย้งแบบเดิม

แนวคิดนี้อาจดูเป็นเพียงการจำกัดเสรีภาพของการนิยาม แต่แท้จริงแล้วคือจุดเริ่มต้นของ “ความมั่นคงเชิงตรรกะ” ของคณิตศาสตร์ยุคใหม่ เพราะมันทำให้เรามั่นใจได้ว่าทุกสิ่งที่เราพิสูจน์ต่อจากนี้ ล้วนตั้งอยู่บนระบบที่ไม่ขัดแย้งในตัวเอง%\footnote{เป้าหมายสำคัญของการนิยามสิ่งต่าง ๆ ในคณิตศาสตร์} 
และนี่คือเหตุผลว่าทำไมในการศึกษาวิชาเซต เราจึงไม่เริ่มต้นด้วยนิยามของคำว่าเซต แต่เริ่มต้นจากสัจพจน์ที่กำหนดขอบเขตของสิ่งที่เราจะเรียกว่าเซตแทน

\newpage
\subsection{เซตว่าง}
แต่ก่อนอื่น จะขอเริ่มจากเซตที่เป็นเซตตั้งต้นและมั่นใจว่าเป็นเซตแน่นอนซึ่งคือ เซตว่าง (empty set) ที่มีสัจพจน์ของเซตว่างกล่าวว่ามีเซตหนึ่งเซตที่ไม่มีสมาชิกเลย โดยให้สัญลักษณ์ของเซตว่างคือ $\varnothing$ กล่าวคือประโยค $x\in\varnothing$ จะเป็นเท็จเสมอไม่ว่าจะกล่าวถึงสมาชิก $x$ ใด ๆ ก็ตาม (จะสังเกตว่าเราไม่ได้นิยามว่าเซตว่างคืออะไร แต่เป็นเซตที่ถูกอธิบายปากเปล่าว่าคืออะไร และมีตัวตนโดยอาศัยสัจพจน์ไม่ใช่การนิยามทางคณิตศาสตร์ขึ้นมา)

\begin{prop}{ความเป็นเอกลักษณ์ของเซตว่าง}{}
	ถ้า $E$ เป็นเซตที่ไม่มีสมาชิกเลย (คือ ไม่มี $x$ ใดที่ $x \in E$) แล้วจะได้ว่า $E = \varnothing$
\end{prop}
%\begin{proof}
%	จะแสดงว่า $E$ และ $\varnothing$ มีสมาชิกตรงกันทุกตัว (พิสูจน์แบบสองทางของการเป็นเซตย่อย)
%	
%	\textbf{(1) พิสูจน์ว่า $E \subseteq \varnothing$}  
%	ให้ $x$ เป็นวัตถุทั่วไปและสมมติว่า $x \in E$ ต้องแสดงว่า $x \in \varnothing$  
%	แต่จากเงื่อนไขของโจทย์ $E$ ไม่มีสมาชิกใด ๆ จึงไม่มี $x$ ใดที่ $x \in E$  
%	ดังนั้นข้อความเชิงอิมพลิเคชัน ``ถ้า $x \in E$ แล้ว $x \in \varnothing$'' เป็นจริงโดยปริยายสำหรับทุก $x$  
%	จึงได้ว่า $E \subseteq \varnothing$
%	
%	\textbf{(2) พิสูจน์ว่า $\varnothing \subseteq E$}  
%	จากข้อก่อนหน้า (เซตว่างเป็นเซตย่อยของทุกเซต) ได้โดยตรงว่า $\varnothing \subseteq E$
%	
%	เมื่อได้ทั้งสองด้าน จึงสรุปว่า $E = \varnothing$
%\end{proof}
\newpage
\subsection{เซตย่อยและเซตกำลัง}
\begin{defn}{เซตย่อย}{}
	กำหนดให้ $A$ และ $B$ เป็นเซต เราจะกล่าวว่า $A$ เป็นเซตย่อยของ $B$ ถ้าทุก ๆ สมาชิกของ $A$ จะต้องเป็นสมาชิกของ $B$ ด้วย กล่าวคือ
	$$
	A\subseteq B \iff \underline{\hspace{6cm}}
	$$
\end{defn}

\begin{exam}
	กำหนดเซต $A = \{1,2,3,4,5\}$
	จงยกตัวอย่างเซตที่เป็นเซตย่อยของ $A$ และเซตที่ไม่ใช่เซตย่อยของ $A$ มาอย่างละ 1 เซตพร้อมทั้งพิสูจน์ให้เห็นจากนิยาม
\end{exam}
\vspace{5cm}

\begin{exer}
	{}{}
	ให้ $A = \{ x\in \N \mid x < 7\}$ and $B = \{ x\in \N \mid x^2 < 49\}$
	จงพิสูจน์ว่า $B\subseteq A$ และพิสูจน์ว่า $A \nsubseteq B$
\end{exer}
\vspace{5cm}

\newpage
\begin{exer}
	{สมบัติถ่ายทอดของเซตย่อย}{}
	กำหนดให้ $A, B$ และ $C$ เป็นเซตใด ๆ จงพิสูจน์ว่า ถ้า $A \subseteq B$ และ $B \subseteq C$ แล้ว $A \subseteq C$
\end{exer}
\vspace{7cm}

\begin{exer}
	{เซตว่างเแ็นเซตย่อยของทุก ๆ เซต}{}
	กำหนดให้ $A$ เป็นเซตใด ๆ จงพิสูจน์ว่า ถ้า $\varnothing \subseteq A$
\end{exer}
\vspace{5cm}

\newpage
\begin{defn}{เซตกำลัง}{}
	กำหนดให้ $A$ เป็นเซต เราจะนิยามเซตกำลัง (power set) ของ $A$ ว่าเป็นเซตของเซตย่อยทั้งหมดของเซต $A$
	$$
	\mathcal{P}(A) := \{\underline{\hspace{6cm}}\}
	$$
\end{defn}
ตัวอย่างเช่น ถ้าให้ $A = \{1,2,3\}$ จะได้ว่า $$\mathcal{P}(A) := \{\underline{\hspace{8cm}}\}$$
\begin{exer}
	{}{}
	ถ้าให้ $A = \{1,2,3\}$
	จงพิสูจน์ว่า $\{1,3\}\}\in\mathcal{P}(A)$ และพิสูจน์ว่า $\{0,3\}\}\notin\mathcal{P}(A)$
\end{exer}
\vspace{2cm}

\newpage
\begin{exer}
	{จำนวนเซตย่อยทั้งหมด}{}
	สำหรับเซต $A$ ใด ๆ จงพิสูจน์ว่าจำนวนเซตย่อยทั้งหมด
	$$
	|\mathcal{P}(A)| = 2^{|A|}
	$$
	คำใบ้: ลองตั้งข้อสังเกตการสร้างแบบเวียนเกิด เพื่อพิสูจน์อุปนัย
\end{exer}

\newpage
\subsection{การดำเนินการของเซต}
\begin{defn}{เซตกำลัง}{}
	กำหนดให้ $A$ และ $B$ เป็นเซต การดำเนินการของเซตต่าง ๆ นิยามเป็นภาษาพูดดังนี้
	\begin{itemize}
		\item Complement: $A'$ คือเซตของสมาชิกในเอกภพสัมพัทธ์ที่ไม่อยู่ใน $A$
		\item Union: $A \cup B$ คือเซตที่มีสมาชิกจาก $A$ หรือ $B$ หรือทั้งคู่
		\item Intersection: $A\cap B$ คือเซตที่ของสมาชิกร่วมกันระหว่างเซต $A$ และเซต $B$
	\end{itemize}
	ซึ่งสามารถเขียนเป็นภาษาคณิตศาสตร์ได้ดังนี้
	\begin{align*}
		A' &= \{\underline{\hspace{6cm}}\}\\
		A\cup B &= \{\underline{\hspace{6cm}}\}\\
		A\cap B &= \{\underline{\hspace{6cm}}\}
	\end{align*}
\end{defn}

\begin{exam}
	{ตัวอย่างคำนวณ}{}
	กำหนดให้ $A = \{1,2,3,4,5,6\}$, $B = \{2,4,6\}$, $C=\{5,6,7,8\}$ และ $D=\{7,8,9\}$ โดยที่เอกภพสัมพัทธ์คือ $\mathcal{U}=\{1,2,\dots,10\}$ จงหา
	\begin{enumerate}
		\item $A\cup B$
		\item $A\cap B$
		\item $B\cap C$
		\item $A\cap D$
		\item $(B\cup C)'$
		\item $(D\cap C') \cup (A\cap B)'$
		\item $\varnothing\cup C$
		\item $\varnothing\cap C$
	\end{enumerate}
\end{exam}
\newpage
\begin{exer}
	Prove that if $A\subseteq B$, then $A\cup B = B$ and $A\cap B = A$.
\end{exer}
\vspace{7.5cm}
\begin{exer}
	Prove that $(A\cap B)' = A' \cup B'$.
\end{exer}
\vspace{7cm}
\newpage
\begin{exer}
	Prove that $\varnothing\cup C = C$ และ $\varnothing\cap C=\varnothing$.
\end{exer}
\vspace{7.5cm}
\begin{exer}
	Prove that $A\cup (B \cap C) = (A\cup B) \cap (A\cup C)$.
\end{exer}
\vspace{7cm}
\newpage
\section{ความสัมพันธ์}
\subsection{คู่อันดับ ผลคูณคาร์ทีเซียน และความสัมพันธ์}
\subsection{ความสัมพันธ์ประเภทต่าง ๆ}
\subsection{ความสัมพันธ์สมมูล และชั้นสมมูล}

\section{ฟังก์ชัน}
\subsection{ฟังก์ชัน โดเมน และเรนจ์}
\subsection{ประเภทของฟังก์ชัน}
\subsection{ฟังก์ชันประกอบ}

\section{ทฤษฎีเซตเชิงการนับ}
\subsection{การสมมูลกันเชิงการนับของเซต และคาร์ดินอลของเซต}
\subsection{Cantor's Theorem}