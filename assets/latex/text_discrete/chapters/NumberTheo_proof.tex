\proofpart
\paragraph{บทพิสูจน์ของ Exercise \ref{divProp}}
\begin{proof}
	content...
\end{proof}

\paragraph{บทพิสูจน์ของคุณสมบัติ \ref{divProp2}}
\begin{proof}
	content...
\end{proof}

\paragraph{บทพิสูจน์ของบทตั้ง \ref{divAlgoLem}}
\begin{proof}
	เราจะพิสูจน์การมี $q$ และ $r$ ด้วยการทำอุปนัยบนตั้วแปรจำนวนเต็ม $m\geq 0$ และ $n>0$ (ทำไม?:แบบฝึกหัด \ref{quoRemExis}) และหลังจากที่พิสูจน์การมีแล้ว เราจะพิสูจน์การมีหนึ่งเดียวในลำดับต่อไป
	
	\textbf{พิสูจน์การมี}
	เมื่อกำหนดให้ $m = 0$ (ขั้นฐานของ $m$) ซึ่งกรณีนี้เป็นกรณีที่ง่ายสำหรับทุก ๆ $n$ เพราะ $0 = n\times 0 + 0$ นั่นคือเราสามารถพิสูจน์ขั้นฐานของ $m$ ได้แล้ว ต่อไปเราจะพิสูจน์ขั้นอุปนัยของ $m$ กัน
	
	พิจารณากรณีที่ $m > 0$ สมมติให้สิ่งที่เราพิจารณากันอยู่ เป็นจริงสำหรับ $m$ กล่าวคือสำหรับทุก ๆ $n>0$ จะมีจำนวนเต็ม $q$ และ $r$ โดยที่ $0\leq r < n$ ที่ทำให้ $m = nq + r$ และเรากำลังจะพิสูจน์สำหรับกรณี $m+1$ โดยที่เราจะแยกพิจารณาตามเศษการหารเป็น 2 กรณี\footnote{เพราะการบวก 1 เพิ่มให้ $m$ กลายเป็น $m+1$ จะกระทบกับเศษ $n-1$ ที่จะกลายเป็น $n$ ซึ่งเป็นเศษการหารของตัวหาร $n$ ไม่ได้} ดังนี้ (1) ถ้า $0\leq r\leq n-2$ และ (2) ถ้า $r= n-1$
	
	กรณีที่ 1) $0\leq r\leq n-2$: จะได้ว่า $m+1 = nq+r+1 = nq + (r+1)$ โดยที่ $0<0+1\leq r+1 \leq n-2 +1 = n-1$ กล่าวคือ มีผลหาร $q$ เดิม และมี $r+1$ เป็นเศษการหาร
	
	กรณีที่ 2) $r= n-1$: จะได้ว่า $m+1 = nq+r+1 = nq+n-1+1=nq+n=n(q+1) + 0$ กล่าวคือ มี $q+1$ เป็นผลหาร และเหลือเศษการหารเป็น $0$ ซึ่งสอดคล้องเงื่อนไขการหารแน่นอน
	
	โดยอุปนัยเชิงคณิตศาสตร์ จึงสรุปได้ว่าสำหรับจำนวนนับ $m$ ใด ๆ และสำหรับจำนวนเต็มบวก $n$ ใด ๆ จะมี $q$ และ $r$ ที่ทำให้ $m=nq + r$ โดยที่ $0\leq r < n$ และในลำดับถัดไป เราจะพิสูจน์การมีหนึ่งเดียวกัน
	
	\textbf{พิสูจน์การมีเพียงหนึ่งเดียว} กำหนดให้มีจำนวนเต็ม $q'$ และ $r'$ อีกชุดที่ทำให้ $m=nq' + r'$ โดยที่ $0\leq r' < n$ กล่าวคือ $ nq + r = nq' + r'$ ซึ่งจะได้ว่า $n(q-q') = r' - r$
%	ซึ่งสรุปได้ว่า $n|r'-r$ และจากคุณสมบัติ \ref{divProp2} ข้อ \ref{divProp2:lessthan} จะได้ว่า $n \leq |r'-r|$ (เพราะ $n$ เป็นจำนวนเต็มบวก จึงได้ว่า $|n|=n$)
	แต่เนื่องจาก $r,r'\in\{0,1,\dots,n-1\}$ จะได้ว่า $0\leq|r'-r|<n$ ทำให้ได้ว่า $0\leq n|q'-q|<n$ จึงสรุปได้ว่า $|q'-q| = 0$ กล่าวคือ $q=q'$ และยังทำให้ได้ตามมาว่า $r'-r = n(q-q') = n\times 0 = 0$ จึงได้ว่า $r=r'$
\end{proof}




\section{Theory Exercise}
\begin{enumerate}
	\item\label{quoRemExis} (คำถามต่อเนื่องจากพิสูจน์ของบทตั้ง \ref{divAlgoLem}) สำหรับจำนวนเต็ม $m\geq 0$ และ $n>0$ ซึ่ง $m = nq + r$ โดยที่ $0\leq r < |n|$ จงพิสูจน์ว่าจะมีจำนวนเต็ม $q'$ และ $r'$ โดยที่ $0\leq r' < |n|$ ที่ทำให้ $-m = nq' + r'$ (และพิสูจน์ในทำนองเดียวกันกับ $m = (-n)q' + r'$ และ $-m = (-n)q' + r'$)
	
	\item จงพิสูจน์บทตั้ง \ref{divAlgoLem} ส่วนการมีโดยใช้หลักการการจัดอันดับดี 
\end{enumerate}