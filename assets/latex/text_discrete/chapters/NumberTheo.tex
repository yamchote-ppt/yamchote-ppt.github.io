\chapter{Number Theory}
\theopart
ทฤษฎีจำนวนเป็นหัวข้อที่จะได้ศึกษาเกี่ยวกับคุณสมบัติของจำนวนเต็มที่เกี่ยวข้องกับการหารลงตัวและตัวประกอบ โดยจะเริ่มศึกษาจากการหารลงตัวก่อน แล้วจึงนำไปนิยามจำนวนประกอบและจำนวนเฉพาะ และนำไปสู่ทฤษฎีสำคัญที่เรียกว่า Fundamental Theorem of Arithmetic ซึ่งพูดถึงการแยกตัวประกอบของจำนวนประกอบด้วยจำนวนเฉพาะซึ่งเป็นทฤษฎีสำคัญที่ทำให้เราสามารถศึกษาคุณสมบัติต่าง ๆ ของจำนวนประกอบได้ เช่นจำนวนของตัวประกอบ และการตรวจสอบการเป็นจำนวนเฉพาะ

และหลังจากที่ศึกษาเกี่ยวกับคุณสมบัติของจำนวน เราจะพูดถึงความสัมพันธ์ของสองจำนวน โดยเริ่มที่การนิยามการหารของจำนวนเต็ม แล้วนำไปสู่เรื่องตัวหารร่วมมากและตัวคูณร่วมน้อยเพื่อศึกษาการมีตัวประกอบร่วมกันของจำนวนตั้งแต่สองจำนวนเป็นต้นไป และจบด้วยเรื่องการสมภาคที่เกี่ยวข้องกับระบบของเศษเหลือ รวมไปถึงการนำไปประยุกต์ใช้ในวิทยาการการเข้ารหัส (cryptography)

โดยทั่วไปแล้ว หัวข้อนี้มักจะถูกใช้เป็นหัวข้อเพื่อฝึกเขียนพิสูจน์ทางคณิตศาสตร์ในรายวิชาที่เรียนเกี่ยวกับพื้นฐานการเขียนพิสูจน์หรือการให้เหตุผลทางคณิตศาสตร์\footnote{เช่นเด็กหลักสูตรคณิตศาสตร์จะมีเรียนวิชา Principle of Mathematics หรือเด็กหลักสูตรวิทยาการคอมพิวเตอร์ก็จะมีวิชา Discrete Mathematics เป็นรายวิชาดังกล่าว} เพราะเป็นหัวข้อที่ทำความเข้าใจนิยามหรือคุณสมบัติได้ง่าย อีกทั้งเป็นสิ่งที่ผู้เรียนคุ้นเคยกันมาตั้งแต่สมัยเด็ก (อย่างน้อยทุกคนที่เปิดอ่านหนังสือเล่มนี้น่าจะเคยเรียนวิธีการตั้งหารยาวเพื่อหาผลหารและเศษมาก่อน)
เลยทำให้ผู้เรียนสามารถมุ่งความสนใจไปที่วิธีการให้เหตุผลทางคณิตศาสตร์ได้มากกว่า แทนที่จะต้องมาทั้งทำความเข้าใจนิยามที่บางครั้งก็ซับซ้อน และต้องฝึกให้เหตุผลไปพร้อมกัน จึงเป็นการดีที่ผู้อ่านที่ยังไม่คุ้นเคยการให้เหตุผลทางคณิตศาสตร์ จะใช้บทนี้เป็นแบบฝึกหัดในการเขียนพิสูจน์

\section{การหารลงตัว}
เราจะเริ่มจากแนวคิดพื้นฐานที่สุดของทฤษฎีจำนวนซึ่งคือ \textbf{การหารลงตัว} ซึ่งถ้าย้อนกลับไปในวัยเด็ก เราจะเริ่มจากการเรียนรู้การหารจำนวนเต็มโดยจดจำวิธีการตั้งหารทั้งวิธีหารสั้นและหารยาวเพื่อให้เราหาผลหารและเศษการหารกันได้เป็น โดยที่เราไม่ได้สนใจว่าจริง ๆ แล้วการหารคืออะไรกันแน่ เพียงแต่มองในมุมมองเชิงการคำนวณว่าคือการแบ่งของ

ทั้งนี้ ถ้าจะต้องการศึกษาเกี่ยวกับการหารลงตัวในรูปแบบทางคณิตศาสตร์ ก็คงไม่สะดวกนักถ้าจะบอกว่าเราหารลงตัวถ้าตั้งหารยาวหรือหารสั้นออกมาแล้วได้เศษเป็น 0 เราจึงจำเป็นที่จะต้องนิยามการหารลงตัวในรูปแบบที่สามารถนำไปใช้พิสูจน์คุณสมบัติต่าง ๆ ต่อได้ง่าย โดยเราจะเห็นว่าเพียงแค่มองมุมกลับกัน จากการถามว่ามีส้ม 10 ผล แบ่งให้คน 5 คนจะได้คนละกี่ผล (มองแบบการหาร) เป็นการมองว่า ถ้าเรามีคน 5 คน และแต่ละคนได้รับส้มไป $x$ ผล แล้วต้องใช้ส้ม 10 ผล ซึ่งเราเปลี่ยนรูปแบบประโยคได้เป็น $5x = 10$ ซึ่งถ้ามีจำนวนส้ม $x$ ผลดังกล่าวที่ทำให้เราสามารถแบ่งส้มกันได้ลงตัวพอดี เราก็จะกล่าวว่า 10 หารด้วย 5 ลงตัวนั่นเอง ทั้งนี้ จะพบว่าหลักสำคัญของการพิจารณาการหารลงตัวก็คือการหา $x$ ดังกล่าวนั่นเอง

ในทำนองเดียวกัน เพียงแต่พิจารณาในกรณีทั่วไป เราจะนิยามการหารลงตัวได้ดังนี้

\begin{defn}{Divisibility}{}
	กำหนดให้ $m$ และ $n$ เป็นจำนวนเต็ม เราจะกล่าวว่า $m$ หารด้วย $n$ ลงตัวก็ต่อเมื่อมีจำนวนเต็ม $k$ ที่ทำให้ $m = nk$ และเขียนแทนด้วยสัญลักษณ์ $n|m$
\end{defn}

จากตัวอย่างด้านบน เราจะกล่าวได้ว่า $5|10$ เพราะเราสามารถให้ส้มคนละ 2 ผลได้ เพื่อแบ่งส้ม 10 ผลให้ 5 คนได้พอดี นั่นคือ $k = 2$ นั่นเองที่ทำให้ $10 = 5\times 2$
~
\boxrule{คำเตือน}{
ในครั้งนี้จะยังคงขอเตือนเรื่องตัวบ่งปริมาณการมีอีกสักรอบ ว่าการที่เราทราบว่า $n|m$ นั้น เราเพียงแค่ทราบว่าเรามี $k$ สักตัวหนึ่งที่ทำให้สมการ $m = nk$ เป็นจริง เพียงแต่ในการเขียนพิสูจน์ที่หลาย ๆ อย่างเป็นตัวแปรไม่ทราบค่า เราจะไม่สามารถระบุค่าของตัวแปร $k$ ที่เกิดขึ้นมาจากการอ้างเหตุผลของการหารลงตัวได้ เราทราบเพียงแค่ว่า $m = nk$ (หรือทดไว้ในหัวเท่านั้นว่าจริง ๆ มันก็คือ $\frac{m}{n}$ แต่เขียนไม่ได้ในทฤษฎีจำนวน) แล้วนำค่า $k$ นี้ไปใช้งานต่อในส่วนอื่น ๆ ของบทพิสูจน์

ในทางกลับกัน แต่ถ้าจะต้องการให้เหตุผลเพื่อสรุปการหารลงตัว สิ่งที่เราต้องทำคือการทดหาจำนวนเต็มสักตัวหนึ่ง (อาจจะเป็นตัวเลขหรือกลุ่มของตัวแปรก็ได้) ที่เมื่อนำมาแทนที่ไว้ในตำแหน่งของ $k$ เพื่อคูณกับ $n$ แล้วได้ผลคูณออกมาเป็น $m$
}

\begin{exam}
	จงพิสูจน์ว่า $25|300$
\end{exam}
\solution{
	จากนิยาม จะเห็นว่าสิ่งที่เราต้องการคือจำนวนเต็มสักจำนวนหนึ่งที่เมื่อนำไปคูณกับ 25 แล้วได้ 300 ซึ่งสามารถคำนวณได้โดยง่ายด้วยการทดเลขแบบเด็ก ๆ $300/25=12$ นั่นคือเราทราบแล้วว่าจำนวนดังกล่าวคือ 25 จะเหลือเพียงแค่นำไปเขียนพิสูจน์
	\begin{proof}
		เพราะ $300 = 25 \times 12$ จึงได้ว่า $25|300$
	\end{proof}
}
	
\begin{exam}
	จงพิสูจน์ว่า $25\nmid310$
\end{exam}
\solution{
	ในทำนองเดียวกัน เราต้องหาจำนวนเต็มสักจำนวนหนึ่งที่เมื่อนำไปคูณกับ 25 แล้วได้ 310 ซึ่งถ้าลองทดเลขคำนวณดูจะพบว่า $310/25 = 12.4$ ซึ่งไม่ใช่จำนวนนับ ดังนั้นเราก็พอจะเดาได้(ถึงแม้จะชัด)ว่าควรที่จะหารไม่ลงตัว ทว่าเหตุผลการหารแล้วไม่เป็นจำนวนเต็มนี้ใช้ในการเขียนพิสูจน์ไม่ได้ เพราะการเขียนพิสูจน์ว่าหารไม่ลงตัว ต้องแสดงว่าไม่ว่าหยิบจำนวนเต็มใดมาคูณกับตัวหารจะไม่ได้ตัวตั้ง
	\begin{proof}
		สมมติให้มีจำนวนเต็ม $n$ ที่ทำให้ $310=25n$ (เรากำลังจะพิสูจน์ด้วยการหาข้อขัดแย้ง)\\
		ซึ่งเราจะเห็นว่า $310 = 25\times 12 + 10$\\
		ดังนั้นจึงได้ว่า
		\begin{align*}
			25n &= 25\times 12 + 10\\
			25n - 25\times 12 &= 10\\
			25(n - 12) &= 10
		\end{align*}
		จากข้อสังเกตว่าถ้า $x$ เป็นจำนวนเต็มที่ $0\leq 25x < 25$ จะได้ว่า $x = 0$\\
		และเพราะ $0\leq 10 = 25(n-12) < 25$ จึงได้ว่า $n-12 = 0$\\
		ดังนั้น จะได้ว่า $10 = 25(n-12) = 25 \times 0 = 0$ ซึ่งเป็นข้อขัดแย้ง\\
		จึงได้ข้อสรุปว่า ไม่มีจำนวนเต็ม $n$ ที่ทำให้ $310 = 25n$
	\end{proof}}
หลังจากที่เรานิยามการหารลงตัวให้สามารถนำไปใช้ในการให้เหตุผลและเขียนพิสูจน์ได้แล้วนั้น(แทนที่จะบอกวิธีการหาผลหารและเศษแบบตั้งหารแล้วดูว่าเศษเป็นศูนย์หรือไม่) เราจะมาเริ่มศึกษาคุณสมบัติต่าง ๆ ของการหารลงตัวกันบ้าง ซึ่งการหารลงตัวเป็นความสัมพันธ์บนจำนวนเต็ม ดังนั้นเราจะเริ่มจากพิจารณากันก่อนว่าคุณสมบัติใดของความสัมพันธ์ที่ความสัมพันธ์การหารลงตัวสอดคล้องบ้าง



\begin{exer}\label{divProp}
	จงเขียนประโยคที่กล่าวถึงคุณสมบัติเชิงความสัมพันธ์ของการหารลงตัวตารางนี้ และพิจารณาว่าจริงหรือไม่ ถ้าจริงจงพิสูจน์ (ดูเฉลยได้ใน Proof Part) แต่ถ้าไม่จริงจงยกตัวอย่างค้าน\\
	\begin{tabular}{|c|c|c|c|c|}
		\hline
		คุณสมบัติ        & นิยาม                 & เขียนโดยใช้การหารลงตัว & จริง & ไม่จริง \\ \hline
		สะท้อน           & $\forall x, xRx$ &                        &               &                           \\ \hline
		ถ่ายทอด          & $\forall x\forall y \forall z, xRy \wedge yRz \rightarrow xRz$  &                        &               &                           \\ \hline
		สมมาตร & $\forall x \forall y, xRy \rightarrow yRx$                      &                        &               &                           \\ \hline
		อสมมาตร          & $\forall x \forall y, xRy \rightarrow \neg y Rx$                       &                        &               &                           \\ \hline
		ปฏิสมมาตร        &  $\forall x \forall y, xRy \wedge yRx \rightarrow x = y$                    &                        &               &                           \\ \hline
	\end{tabular}
\end{exer}
\solution{...}

นอกจากนั้น เรายังได้คุณสมบัติต่าง ๆ ดังต่อไปนี้
\begin{prop}[label=divProp2]{คุณสมบัติการหารลงตัว}{} 
	กำหนดให้ $m,n,p$ เป็นจำนวนเต็มใด ๆ จะได้ว่า
	\begin{enumerate}[itemsep=0mm]
		\item $1|m$ และ $m|m$
		\item ถ้า $m\neq 0$ แล้ว $m|0$
		\item ถ้า $m|n$ แล้ว $m|np$
		\item ถ้า $p\neq 0$ และ $m|n$ แล้ว $pm|pn$
		\item ถ้า $m|n$ และ $m|p$ แล้ว $m|(n+p)$
		\item ถ้า $m|n$ และ $m|p$ แล้ว $m|(xn+yp)$ สำหรับทุก ๆ จำนวนเต็ม $x, y$
		\item ถ้า $m|n$ แล้ว $|m|\leq |n|$ \label{divProp2:lessthan}
	\end{enumerate}
\end{prop}

\paragraph{แนวคิดของทฤษฎีและแนวคิดการเขียนพิสูจน์:}\footnote{ไม่ใช่การเขียนพิสูจน์ เป็นแค่แนวคิด}
\begin{enumerate}
	\item ในข้อนี้ค่อนข้างตรงไปตรงมาเหมือนที่เคยท่องกันตอนเด็ก ๆ ว่า 1 หารทุกจำนวนลงตัว เพราะ 1 คูณอะไรก็ได้ตัวมันเอง กล่าวแบบรัดกุมคือ $1\cdot n = n$ สำหรับทุก ๆ จำนวนเต็ม $n$
	\item และในทำนองเดียวกัน เมื่อเราใช้ 0 เป็นตัวตั้ง เราน่าจะตอบกันได้ทันทีว่า 0 คูณอะไรก็ได้ 0
	\item\label{divMulti} ในข้อนี้นั้น แนวคิดตั้งต้นมาจากการที่เปรียบเสมือนเรามีเศษส่วนที่ตัดกันได้หมดอยู่แล้ว ($\frac{n}{m}$ ตัดกันได้หมด) ต่อให้เราคูณตัวตั้งเพิ่มเข้าไปด้วยอะไร ($p$) ก็ตาม เราก็ควรที่จะยังคงตัดได้ $\frac{np}{m}$ ลงตัวเช่นเดิมด้วยการตัดคู่เดิม ซึ่งถ้าเรามองในแง่การเขียนพิสูจน์ เปรียบเสมือนเรามีจำนวนหนึ่งที่คูณตัวหารได้ตัวตั้งอยู่แล้ว ถ้าสนใจกับตัวตั้งที่เพิ่มขึ้น $p$ เท่า ผลหารก็ควรจะเพิ่มขึ้น $p$ เท่าเช่นกัน ซึ่งเรากล่าวในอีกนัยหนึ่งได้ว่าการหารลงตัวถูกรักษาไว้ภายใต้การคูณตัวตั้ง (divisibility is preserved under numerator multiplication) 
	\item เหมือนการคูณทั้งเศษและส่วนของเศษส่วนที่ยังคงให้ค่าผลหารเท่าเดิมอยู่ $\frac{n}{m} = \frac{pn}{pm}$
	\item\label{divAddi} เปรียบเสมือน $\frac{n+p}{m} = \frac{n}{m} + \frac{p}{m}$ โดยความหมายของคุณสมบัตินี้คือการหารลงตัวยังคงถูกรักษาไว้ภายใต้การบวกของตัวตั้ง
	\item เราเรียกพจน์ $xn + yp$ ว่าผลรวมเชิงเส้น (linear combination) ซึ่งเป็นผลขยายมาจากข้อ \ref{divMulti} และข้อ \ref{divAddi}
\end{enumerate}
สิ่งที่อธิบายในแต่ละข้อ เป็นเพียงแนวคิดเชิงที่มา(การตั้งข้อสังเกต) และแนวคิดเชิงการให้เหตุผล(แนวทางการเขียนพิสูจน์) ไม่ใช่การเขียนพิสูจน์ โดยประเด็นสำคัญที่สุดคือในการเขียนพิสูจน์เราไม่สามารถใช้เศษส่วนในแง่การคำนวณได้ (เช่น $\frac{n}{m} = \frac{pn}{pm}$ เป็นต้น)

\section{ขั้นตอนวิธีการหาร: Division Algorithm}
หัวข้อที่แล้ว เราได้ศึกษาเกี่ยวกับการหารลงตัว หรือการเป็นตัวประกอบของจำนวนเต็มไป แต่ก็จะพบว่าในบางครั้งเราอยากจะอธิบายการหารได้กับทุกคู่ของจำนวนเต็ม กล่าวคือ เราอยากขยายไอเดียการหารให้ทั่วไปมากขึ้น ไม่ได้สนใจเพียงแค่การหารลงตัวหรือไม่ลงตัวที่เป็นคุณสมบัติที่ขึ้นกับจำนวนเต็มที่เป็นตัวตั้งเท่านั้น

และถ้านึกย้อนไปในวัยเด็ก (อีกครั้ง) หลายคนน่าจะจำกันได้ดีว่าพวกเราเริ่มเรียนการหารกันด้วยการตอบผลหารและเศษเหลือจากการหาร แต่สิ่งที่พวกเราได้เรียนกันในวัยเด็ก เป็นเพียงแค่วิธีการเขียนเพื่อให้เราในวัยเด็กที่ยังไม่มีแนวคิดแบบนามธรรมสามารถทำตามได้ กล่าวคือเราถูกคาดหวังเพียงแค่หาคำตอบที่ถูกต้องให้ได้ก่อน แต่ไม่ได้เรียนว่าทำไมทำแบบนั้นถึงทำได้ หรืออะไรคือที่มาของแนวคิด

นอกจากนั้น จะสังเกตว่าวิธีการที่พวกเราได้เรียนโดนจำกัดอยู่แค่จำนวนเต็มบวก
กล่าวคือ ถ้าตัวตั้งหรือตัวหารเป็นจำนวนเต็มลบ เราจะยังคำนวณหาผลหารและเศษกันไม่เป็นอยู่ดี (ตัวอย่างเช่นจงหาผลหารของ -21 หารด้วย 5)
ในครั้งนี้ เราจึงจะนำแนวคิดเรื่องผลหารและเศษเหลือที่คำนวณกันได้เก่งมากกับจำนวนบวก มาเขียนนิยามกันในรูปแบบคณิตศาสตร์ เพื่อให้เราสามารถศึกษาประเด็นที่เกี่ยวกับผลหารและเศษเหลือได้ทั่วไปและเป็นคณิตศาสตร์มากขึ้น

แต่โชคดี! ที่อย่างน้อย พวกเราก็ได้เรียนสิ่งที่เรียกว่าการตรวจสอบผลหารด้วยวิธีการ
$$
\text{ตัวตั้ง} = \text{ตัวหาร}\times\text{ผลหาร} + \text{เศษ}
$$

ซึ่งจริง ๆ แล้ว สิ่งนี้ก็คือนิยามของการหารที่ทำให้พวกเราสามารถนิยามการหารของจำนวนเต็มได้ทั่วไปมากขึ้นด้วยการหาผลหาร และเศษเหลือมาเติมในสมการ แต่ทั้งนี้ ก่อนนิยามสิ่งใด ๆ ก็ตามในคณิตศาสตร์ (เช่นในที่นี้เรากำลังจะนิยามสิ่งที่เรียกว่า ผลหาร และเศษเหลือ) สิ่งหนึ่งที่เราต้องพิจารณากันก่อนก็คือการมีค่าได้จริง (ไม่ใช่พูดได้บ้างไม่ได้บ้าง) กับการมีเพียงหนึ่งเดียว (เพราะกำลังจะตั้งชื่อ: well-defined)

\begin{lem}[label=divAlgoLem]{การมีผลหารและเศษเหลือ}{}
	กำหนดให้ $m$ และ $n$ เป็นจำนวนเต็มใด ๆ โดยที่ $n\neq 0$ จะมีจำนวนเต็ม $q$ และ $r$ เพียงคู่เดียวเท่านั้นที่ทำให้ $m = nq + r$ โดยที่ $0\leq r < |n|$
\end{lem}

\begin{defn}{Division Algorithm}{}
	กำหนดให้ $m$ และ $n$ เป็นจำนวนเต็มใด ๆ โดยที่ $n\neq 0$ แล้ว $q$ และ $r$ จากบทตั้ง \ref{divAlgoLem} ว่าผลหาร (quotient) และเศษเหลือ (remainder) ตามลำดับ
\end{defn}













