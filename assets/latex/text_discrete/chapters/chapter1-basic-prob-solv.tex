\chapter{Fundamental of Problem Solving}

เราจะเริ่มบทแรกของหนังสือเล่มนี้ด้วยทักษะที่สำคัญที่สุดไม่ว่าจะในการเรียนคณิตศาสตร์ หรือจะคอมพิวเตอร์ก็ตาม นั่นคือทักษะการแก้ปัญหา (problem solving) เพราะแก่นแท้ของตัววิชาเหล่านี้นั้นคือการนำความรู้ไปใช้ในการแก้ปัญหาต่าง ๆ ไม่ว่าจะปัญหาในตัววิชาเองในรูปแบบปัญหาเชิงการคำนวณ (computational problem) หรือปัญหาในโลกจริง กล่าวคือ ปัญหาคือสิ่งที่เราจะต้องพบเจอเป็นเรื่องปกติในการเรียนวิชานี้

ในบทนี้เราจะเริ่มจากมาดูกันก่อนว่าปัญหาคืออะไร และการแก้ปัญหาคืออะไร เพราะก่อนจะลงมือแก้ปัญหา เราก็ต้องเข้าใจก่อนว่าสิ่งเหล่านี้คืออะไร หลังจากที่เข้าใจเกี่ยวกับสิ่งที่เรียกว่าปัญหาแล้ว เราจะมาต่อกันว่าทักษะหรือแนวคิดอะไรบ้างที่สำคัญในการแก้ปัญหา โดยจะไม่กล่าวถึงรายละเอียดปลีกย่อยของเทคนิคการแก้ปัญหา เพราะในแต่ละรูปแบบปัญหาที่ต่างกัน ก็จะมีรายละเอียดในเรื่องวิธีการแก้ปัญหาหรือเทคนิคการแก้ปัญหาที่แตกต่างกันออกไป เหมือนการทำโจทย์คณิตศาสตร์ที่รูปแบบโจทย์ที่แตกต่างกันก็อาจจะมีเทคนิคที่แตกต่างกัน แต่ว่าสิ่งที่จะทำให้เรารู้ว่าต้องใช้เทคนิคหรือวิธีการอะไรในการแก้ปัญหาที่ต้องการแก้ก็คือประสบการณ์ที่เราจะได้ฝึกกันในแต่ละบท ๆ ต่อจากนี้นั่นเอง

\section{Problem Solving คืออะไร}

ก่อนจะถามว่าการแก้ปัญหาคืออะไร ก็คงไม่เสียเวลาอะไรนักถ้าเราจะมาพูดคุยตกลงกันให้เข้าใจก่อนว่า อะไรคือ\textbf{ปัญหา} ซึ่งถ้าเราเปิดดูความหมายตามราชบัณฑิต คำนี้จะมีความหมายว่า
\begin{center}
	น. ข้อสงสัย, ข้อขัดข้อง, เช่น ทำได้โดยไม่มีปัญหา, คำถาม, ข้อที่ควรถาม, เช่น ตอบปัญหา, ข้อที่ต้องพิจารณาแก้ไข เช่น ปัญหาเฉพาะหน้า ปัญหาทางการเมือง.
\end{center}
ซึ่งบางความหมาย อาจจะรู้สึกว่าปัญหาก็คืออะไรที่รู้สึกว่าไม่ดี เพราะจะทำให้สิ่งต่าง ๆ ดำเนินไปไม่เป็นไปตามที่ควรจะเป็น เช่นข้อขัดข้อง หรือข้อที่ต้องพิจารณาแก้ไข ทว่ายังมีความหมายอีกกลุ่มหนึ่งที่ดูน่าสนใจคือ ข้อสงสัย ข้อควรถาม ที่เรามักพูดกันว่า ``ตอบปัญหา''

ในหนังสือเล่มนี้ (และในคณิตศาสตร์ รวมไปถึงการเขียนโปรแกรมคอมพิวเตอร์) เราจะให้ความหมายของ \textbf{ปัญหา} คือ โจทย์ที่ถามหรือกล่าวขึ้นมาพื่อต้องการคำตอบโดยอาจจะมีเงื่อนไขบางอย่างหรือไม่มีก็ได้ โดยจะเป็นการกล่าวถึงสถานการณ์ที่มีสิ่งตั้งต้นอะไรสักอย่าง แล้วสุดท้าย(หลังจากผ่านกระบวนการอะไรสักอย่าง)จะได้สิ่งที่ต้องการออกมา

ตัวอย่างเช่น ``บริษัทจัดสรรแม่บ้านทำความสะอาดตามสั่งแห่งหนึ่งได้รับการจองคิวใช้บริการแม่บ้านเข้ามาจำนวนหนึ่งจากลูกค้าหลายราย โดยที่ลูกค้าแต่ละคนก็มีจำนวนวันที่ต้องการใช้บริการแม่บ้านไม่เหมือนกัน ทางบริษัทเลยอยากรู้ว่าต้องเตรียมแม่บ้านไว้กี่คน'' ซึ่งเราจะพบว่าปัญหานี้เราต้องการรู้ว่าต้องเตรียมแม่บ้านไว้กี่คน โดยเรามีรายการการจองคิวเป็นตัวตั้งของการตอบปัญหานี้

จากตัวอย่างที่กล่าวมา จะเรียกสิ่งตั้งต้น (เช่นรายการการจองคิวที่บริษัทได้รับ) ว่า\textbf{ข้อมูลขาเข้า} (input) และเราจะเรียกสิ่งที่ได้ออกมา (เช่นจำนวนแม่บ้านที่ต้องเตรียมไว้) ว่า\textbf{ข้อมูลขาออก} (output) ดังนั้น เราอาจจะกล่าวได้อีกแบบหนึ่งว่าปัญหาก็คือการมีข้อมูลขาเข้า และข้อมูลขาออกที่ต้องการ และสิ่งที่เราต้องลงแรงหาก็คือ วิธีการที่จะแปลเปลี่ยนข้อมูลขาเข้าดังกล่าวให้ได้ข้อมูลขาออกตามที่ต้องการ ซึ่งเราจะเรียกกระบวนการการหาวิธีการดังกล่าวว่า\textbf{การแก้ปัญหา} (problem solving) และจะเห็นว่าสิ่งสำคัญอันดับแรกสุดไม่ว่าเราจะแก้ปัญหาอะไรก็ตามคือการทำความเข้าใจภาพรวมของโจทย์ (problem statement) ว่าตัวปัญหาคืออะไร และระบุให้ได้ว่าอะไรคือข้อมูลขาเข้า และข้อมูลขาออก โดยถ้าเทียบกับตัวอย่างบริษัทแม่บ้านทำความสะอาดก่อนหน้า จะมีรายละเอียดดังนี้

\begin{itemize}
	\itemsep0em 
	\item \textbf{โจทย์}: หาวิธีการในการคำนวณจำนวนแม่บ้านที่ต้องเตรียมไว้เมื่อได้รับรายการการจองคิวใช้บริการจากลูกค้า
	\item \textbf{ข้อมูลขาเข้า}: รายการการจองคิวใช้บริการ
	\item \textbf{ข้อมูลขาออก}: จำนวนแม่บ้านที่ต้องเตรียมไว้
\end{itemize}

ทั้งนี้ ตัวปัญหาเองก็อาจจะถูกแบ่งกลุ่มออกเป็นประเภทต่าง ๆ ได้หลายประเภท แต่ปัญหาที่เราจะสนใจกันในหนังสือเล่มนี้นั้นจะเป็นปัญหาในกลุ่ม\textbf{ปัญหาเชิงการคำนวณ} (computational problem) หรือหนังสือบางเล่มจะเรียกว่าปัญหาเชิงการประมวลผล ซึ่งคำว่าคำนวณในที่นี่ไม่ได้หมายถึงเพียงแค่การบวก ลบ คูณ หาร หรือการทำโจทย์คณิตศาสตร์ (calculation) แต่ยังรวมไปถึงการวางแผนเชิงกระบวนการ เชิงตรรกะ เชิงเหตุผล หรือรวมไปถึงการคิดเชิงสัญลักษณ์เองก็ด้วย ไม่จำเป็นว่าจะต้องเป็นปัญหาที่เกี่ยวกับตัวเลขเพียงเท่านั้น ซึ่งกระบวนการการแก้ปัญหาเชิงการคำนวณถือว่าเป็นทักษะที่สำคัญที่สุดในการเขียนโปรแกรม รวมไปถึงการศึกษาคณิตศาสตร์ และวิทยาการคอมพิวเตอร์ โดยเราจะได้กล่าวถึงรายละเอียดของกระบวนการดังกล่าวในหัวข้อถัดไป

\section{การแก้ปัญหาเชิงการคำนวณ}

จากหัวข้อที่แล้ว เราอาจกล่าวโดยสรุปได้ว่าปัญหาเชิงการคำนวณก็คือปัญหาที่จะสามารถแก้ได้ด้วยคอมพิวเตอร์โดยการออกแบบอัลกอริทึมที่เหมาะสม และในการแก้ปัญหาเชิงการคำนวณนั้น จะมีทักษะที่สำคัญที่จะช่วยให้เราแก้ปัญหาเชิงการคำนวณได้อย่างมีประสิทธิภาพอยู่ 4 ทักษะได้แก่
\begin{enumerate}
	\itemsep0em 
	\item การแบ่งย่อยปัญหา (decomposition)
	\item การเข้าใจรูปแบบ (pattern recognition)
	\item การคิดเชิงนามธรรม (abstraction)
	\item การออกแบบขั้นตอนวิธี (algorithm design)
\end{enumerate}

\subsection{การแบ่งย่อยปัญหา (decomposition)}
ในการแก้ปัญหาหนึ่งที่เราได้รับมานั้น อาจเป็นการยากถ้าเราจะหาวิธีที่แปลงข้อมูลขาเข้าให้กลายเป็นข้อมูลขาออกได้ภายในขั้นเดียว อาจจะเนื่องมาจากการแก้ปัญหาดังกล่าวต้องการขั้นตอนย่อย ๆ หรือเครื่องมือย่อย ๆ ในการแก้ปัญหานั้น ดังนั้นเราจึงควรย่อยปัญหาใหญ่ให้ออกเป็นปัญหาย่อย ๆ ที่จะสามารถแก้ได้ง่าย ๆ ไม่ซับซ้อนก่อน

ตัวอย่างเช่นเราอยากจะต่อจิกซอว์สักรูปหนึ่ง คงเป็นการยากถ้าเราจะเทจิกซอว์ทั้งหมดลงมาในแผ่นเดียวแล้วต่อขึ้นมาด้วยการมองภาพทั้งภาพในเวลาเดียวกัน แต่คงจะดีขึ้นถ้าเรารู้ว่าในภาพมีองค์ประกอบย่อย ๆ ที่เห็นความแตกต่างเรื่องสีอย่างชัดเจน เช่นมีบริเวณหนึ่งที่มีแต่สีแดง และมีอีกบริเวณหนึ่งที่มีแต่สีเขียว หรืออีกบริเวณหนึ่งเป็นลายผ้าสีเหลืองลายจุดสีส้ม เราก็เลยจะแบ่งปัญหาการต่อจิกซอว์ทั้งผืนเป็นปัญหาการต่อจิกซอว์กลุ่มย่อย ๆ ที่เป็นสีแดง, ปัญหาการต่อจิกซอว์กลุ่มย่อย ๆ ที่เป็นสีเขียว และ ปัญหาการต่อจิกซอว์กลุ่มย่อย ๆ ที่เป็นสีเหลืองลายจุดสีส้ม ซึ่งจะทำให้เกิดปัญหาที่เล็กลงและอาจจะซับซ้อนน้อยลงเพราะเรากำจัดตัวเลือกจิกซอว์ที่ไม่เกี่ยวข้องกับบริเวณดังกล่าวออกไปได้เยอะ

ขออีกสักตัวอย่างที่ดูเป็นปัญหาเชิงการคิดเลขมากขึ้น เช่นปัญหาการแก้สมการจำนวนเต็ม $x + y + 12z = 30$ โดยที่ $x, y$ และ $z$ เป็นจำนวนเต็มบวกสามจำนวนที่ต่างกัน โดยโจทย์ต้องการว่ามีผลเฉลย $(x,y,z)$ ดังกล่าวทั้งหมดกี่ครูปแบบ ซึ่งแน่นอนว่าถ้าเราไล่ไปเรื่อย ๆ ก็อาจจะเสร็จได้ไม่ได้ยากมาก เพราะเลขเราต้องการผลบอกแค่ 30 ถ้าต้องไล่ 0 ถึง 30 ก็มีอยู่ไม่เกิน $31 \times 31 \times 31 = 29791$ รูปแบบ ซึ่งถ้าให้คอมพิวเตอร์ช่วยรันให้ก็คงใช้เวลาไม่นาน แต่ถ้าใช้คนก็อาจจะเหนื่อยก่อนและมีคิดผิดบ้างได้ แต่เราจะเห็นว่าการเพิ่มขึ้นของค่า $z$ นั้นกลับมีประโยชน์อย่างมาก เพราะเพิ่มขึ้น 1 ค่าในด้านซ้ายจะเพิ่มขึ้นไปถึง 12 ดังนั้นเราจึงอาจจะสังเกตได้ไม่ยากว่าแยกพิจารณาตามค่า $z$ ไปเลยก็ได้ โดยที่ $z = 0,1,2$ (เพราะถ้ามากกว่านี้ ผลบวกจะเกิน 30) กล่าวคือ เราจะแยกปัญหาหลักเราออกเป็นปัญหาย่อย 3 ปํญหาย่อยคือ
\begin{enumerate}
	\itemsep0em 
	\item เมื่อ $z = 0$: แก้สมการ $x + y = 30$
	\item เมื่อ $z = 1$: แก้สมการ $x + y = 18$
	\item เมื่อ $z = 2$: แก้สมการ $x + y = 6$
\end{enumerate}
ซึ่งแต่ละปัญหาย่อย จะสามารถแก้ได้ด้วยการนับง่าย ๆ

ในการแยกปัญหาย่อยนั้น อาจจะได้ปัญหาย่อยมาในรูปแบบที่แยกกันทำ ต่างคนต่างอิสระจากกัน ทำเสร็จแล้วค่อยนำคำตอบของแต่ละปัญหามาผนวกรวมร่างกันให้กลายเป็นปัญหาใหญ่ เช่นตัวอย่างสมการข้างต้นที่เราสามารถแก้ปัญหาไหนก่อนก็ได้ไม่มีผลต่อกัน หรือเราอาจจะได้ปัญหาย่อยที่มาในรูปแบบที่ต้องทำงานต่อเนื่องกันโดยที่เมื่อทำปัญหาย่อยที่ 1 เสร็จให้นำผลของปัญหาย่อยที่ 1 ไปใช้ต่อเป็นข้อมูลขาเข้าของปัญหาย่อยที่ 2 ก็ได้ ทั้งนี้ ไม่มีกฏตายตัวในการตั้งปัญหาย่อย ขึ้นอยู่กับมุมมองต่อปัญหาตรงหน้าของเรา ณ เวลานั้น

\subsection{การเข้าใจรูปแบบ (pattern recognition)}
ใน




