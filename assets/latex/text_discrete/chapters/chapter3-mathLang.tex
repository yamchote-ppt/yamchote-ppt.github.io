\chapter{Mathematics as a Language}\label{chap:mathLang}

บทนี้จะเป็นบทสั้น ๆ เน้นที่การเล่าให้เห็นภาพรวมของคณิตศาสตร์ในรูปแบบการเรียนเพื่อหาเหตุผล เป้าหมายของบทนี้เพียงเพื่อต้องการเปลี่ยนทัศนคติของผู้อ่านบางท่านเกี่ยวกับคณิตศาสตร์ ก่อนที่เราจะลงลึกไปสู่คณิตศาสตร์จริง ๆ ในบทถัด ๆ ไป อย่างน้อยก็อยากให้หลังจากที่อ่านบทนี้จบ ผู้อ่านจะมองว่าคณิตศาสตร์คือวิชาของการอธิบายสิ่งต่าง ๆ ในโลก และการให้เหตุผลของความเป็นไปในสิ่งต่าง ๆ ไม่ใช่แค่การคิดเลข

หลายท่าน (รวมถึงเด็ก ๆ จากประสบการณ์การสอนพิเศษมาหลายปีของผู้เขียน) อาจจะจำความรู้สึกมาจากตอนเรียนระดับมัธยมต้นว่าวิชาคณิตศาสตร์เป็นวิชาที่เกี่ยวกับการคิดเลข จำสูตรไปแทนค่าหาคำตอบ ขอแค่จำสูตรไดเ้ยอะ ๆ อ่านโจทย์แล้วรู้ว่าใช้สูตรไหน คิดเลขให้ไว ๆ ก็น่าจะทำข้อสอบได้คะแนนดีกันแล้ว ผู้เขียนเคยเจอถึงขั้นว่ามีนักเรียนใช้วิธีดูว่าข้อนี้ต้องหยิบสูตรไหนมาคิดโดยการดูว่าเจอคีย์เวิร์ดอะไรในโจทย์ และบอกคนอื่นได้ว่าเราเรียนคณิตศาสตร์รู้เรื่อง แต่ทว่า พอขึ้นมาเรียนในระดับมัธยมปลาย กลับพบว่าคณิตศาสตร์เปลี่ยนไปอย่างมาก เราได้เรียนเรื่องเซต เรื่องตรรกศาสตร์ ความสัมพันธ์และฟังก์ชันในระดับชั้นมัธยมศึกษาปีที่ 4 กันเป็นเรื่องแรก ๆ ที่ตัวเนื้อหาตามหนังสือเรียนนั้น แทบไม่ใช่การคิดเลขเลย แต่เป็นเรื่องของการเรียนรู้การใช้สัญลักษณ์ เรียนรู้การให้เหตุผล เพื่อใช้สื่อสารกันในโลกของคณิตศาสตร์ ซึ่งอาจจะต้องโทษวิธีการสอนของครูมัธยมไทยหลาย ๆ ท่านที่ทำให้เนื้อหาพวกนี้หนีไม่พ้นสอนการคิดเลขเหมือนเดิม เช่น\textbf{จัดรูป}อย่างง่ายของประพจน์ \textbf{คำนวณ}หาผลยูเนียน \textbf{คำนวณ}หาผลอินเตอร์เซคชัน หรือแม้กระทั่ง\textbf{คำนวณ}หาผลค่าความจริงในวิชาตรรกศาสตร์

ในบทนี้จะขอยกบทเรียนที่เป็นตัวละครสำคัญที่ทำให้เรามองคณิตศาสตร์เป็นเรื่องของภาษา แทนที่จะมองว่าเป็นเครื่องมือในการคิดเลขได้แก่ (1) เซต (2) ตรรกศาสตร์ (3) ความสัมพันธ์ และ (4) ฟังก์ชัน ซึ่งเปรียบได้กับเป็น 4 เสาหลักของคณิตศาสตร์เลยก็ว่าได้ (จะมีกล่าวถึงในตรรกศาสตร์อันดับหนึ่ง)

\section{เซต} อย่างเช่นเรื่องเซต เป้าหมายของบทนี้คือการต้องการใช้คณิตศาสตร์อธิบายความเป็นกลุ่ม ความเป็นสมาชิกของสิ่งใดสิ่งหนึ่ง เช่นเราบอกว่านาย ``a เป็นนักเรียน'' เราก็จะมองในรูปแบบคณิตศาสตร์ว่าเรามีเซตของนักเรียน ในที่นี้สมมติให้เป็น $S$ ที่ใครก็ตามที่อยู่ในเซต $S$ จะถูกอธิบายความเป็นนักเรียน และนาย a ก็เป็นสมาชิกในเซตนักเรียน จึงเขียนเป็นสัญลักษณ์แทนประโยคดังกล่าวได้ว่า $\text{a} \in S$ ที่แทนการกล่าวว่า ``a เป็นนักเรียน''

ในทำนองเดียวกัน ถ้าเรากล่าวว่านักเรียนก็เป็นบุคลากรของโรงเรียน ก็เปรียบเสมือนเรามีเซตที่เป็นกลุ่มของบุคลากรของโรงเรียน สมมติให้เป็น $X$ และมีเซตของนักเรียนเป็นกลุ่มย่อยในนั้น หรือกล่าวว่า เซตของนักเรียนเป็นเซตย่อยของเซตบุคลากร โดยเขียนเป็นสัญลักษณ์ว่า $S \subseteq X$

อีกทั้ง ถ้าเรานำนิยามทางคณิตศาสตร์ของการเป็นเซตย่อยมาจับกับประโยคทั้งสอง
\begin{defn}[label=subset0]{เซตย่อย}{} 
	ให้ $A$ และ $B$ เป็นเซต เราจะกล่าวว่า $A$ เป็นเซตย่อยของ $B$ หรือเขียนว่า $A \subseteq B$ ก็ต่อเมื่อ สำหรับทุก $x$ ถ้า $x\in A$ แล้ว $x\in B$
\end{defn}
\noindent ซึ่งเรามีประโยค (1) $\text{a} \in S$ และ (2) $S \subseteq X$ จากนิยามของเซตย่อย \ref{subset0} เราจะเห็นความสอดคล้องระหว่างสิ่งที่เรามีกับเครื่องมือที่เรารู้ดังนี้
\begin{itemize}
	\item $S$ เปรียบเสมือน $A$ ในนิยาม และ $X$ เปรียบเสมือน $ฺB$ ในนิยาม
	\item $\text{a} \in S$ สอดคล้องกับประโยค $x\in A$
	\item $S \subseteq X$ สอดคล้องกับประโยค $A \subseteq B$
\end{itemize}
จากนิยามดังกล่าวทำให้เราสรุปได้ว่า $x\in B$ (ในนิยาม) ซึ่งสอดคล้องกับประโยค $\text{a} \in X $
หรือกล่าวคือ a เป็นบุคลากรของโรงเรียนเช่นกัน

ในบางครั้งนั้น เราต้องการอธิบายเชื่อมโยงกันระหว่าง 2 กลุ่ม (หรือมากกว่า) เช่นเราต้องการอธิบายว่านาย a เป็นนักเรียนที่ลงเรียนวิชาคณิตศาสตร์ดีสครีตและวิชาโครงสร้างข้อมูล ซึ่งเป็นการกล่าวถึงกลุ่มของนักเรียน 2 กลุ่ม คือกลุ่มของนักเรียนที่ลงเรียนวิชาดีสครีต (สมมติให้เป็น $C$) และกลุ่มของนักเรียนที่ลงเรียนวิชาโครงสร้างข้อมูล (สมมติให้เป็น $D$) และชัดเจนว่า $C\subseteq S$ และ $D\subseteq S$ เพราะมีเพียงนักเรียนเท่านั้นที่ลงทะเบียนเรียนได้ กล่าวคือทุกคนที่จะลงทะเบียนเรียนวิชาดังกล่าวได้ต้องเป็นนักเรียน (ลองคิดทิศทางให้ดีว่าเป็น (1) ถ้าลงทะเบียนเรียนแล้วต้องเป็นนักเรียน หรือ (2) ถ้าเป็นนักเรียนแล้วต้องลงทะเบียนเรียน) แต่ทั้งนี้เราจะพูดอธิบายตลอดว่า ``$\text{a}\in C$ และ $\text{a}\in D$'' เพื่อเป็นตัวแทนประโยคดังกล่าวก็คงไม่กระชับมากนัก และดูยังต้องเขียนเป็นประโยค 2 ประโยคมาเชื่อมกัน ไม่ใช่การเขียนประโยคของเซตเลย จึงได้นิยามการเชื่อมการอยู่ร่วมกันทั้ง 2 กลุ่มด้วยการอินเตอร์เซคชัน (intersection) กล่าวคือ $\text{a}\in C\cap D$ ซึ่งจะเห็นว่าจากประโยคที่ตัวหลักคือคำเชื่อม ``และ'' จะถูกเขียนให้อยู่ในรูปของเซตล้วนและตัวหลักของประโยคคือ ``การเป็นสมาชิก'' แทน

เราจะเห็นว่าคำศัพท์ต่าง ๆ ที่เกี่ยวกับเซตนั้น ก็เกิดมาเพื่อใช้ในการอธิบายปรากฏการณ์ที่เกี่ยวข้องกับการเป็นสมาชิกในกลุ่มนั่นเอง
ทว่าสิ่งที่อธิบายในเรื่องของวิธีการสรุปผลในข้างต้นนั้นก็ไม่ใช่บทบาทหน้าที่ของเรื่องเซต เพราะเซตเป็นเพียงการบอกว่ามีใครเป็นสมาชิกบ้าง แต่การสรุปผลต่างๆ เป็นบทบาทหน้าที่ของสิ่งที่เรียกว่า ``ตรรกศาสตร์''

\section{ตรรกศาสตร์}
หรืออย่างในเรื่องตรรกศาสตร์เอง ก็เป็นการเรียนรู้โครงสร้างประโยคในภาษาคณิตศาสตร์ รวมไปถึงการเชื่อมโยงระดับประโยค พร้อมทั้งมีการพิจารณาความเป็นจริงหรือไม่จริงหรือที่เรียกกันว่า ค่าความจริง\footnote{จริง ๆ แล้วยังมีการศึกษาตรรกศาสตร์ในรูปแบบที่เราไม่สนใจเรื่องค่าความจริงด้วย แต่จะสนใจในเรื่องของความถูกต้องของรูปแบบโครงสร้างการเขียน และสรุปผลด้วยโครงสร้างของประโยค ซึ่งเรียกว่าตรรกศาสตร์เชิงวากยสัมพันธ์} เป็นเบื้องหลังของการนิยามอยู่ เพราะตรรกศาสตร์ก็เกิดมาเพื่อต้องการใช้คณิตศาสตร์ในการทำความเข้าใจระบบความคิดของมนุษย์ในรูปแบบที่มาตรฐานขึ้น เลยถูกสร้างเลียนแบบการสื่อสารของมนุษย์ นำภาษามนุษย์มาทำให้เป็นรูปแบบเชิงสัญลักษณ์ พร้อมกับมีการนำไปใช้เพื่อวิเคราะห์ความเป็นเหตุเป็นผลเชิงค่าความจริง

ไม่เพียงแค่พิจารณาค่าความจริงของตัวประโยคเท่านั้น การศึกษาเชิงตรรกศาสตร์เองก็ยังรวมไปถึงการสร้างประโยคเพื่ออธิบายความเป็นตัวตนของสิ่งของในคณิตศาสตร์เช่นกัน เช่น ประโยค ``$x$ เป็นนักเรียน'' (สมมติแทนด้วยสัญลักษณ์ $P(x)$) จะถูกใช้เพื่อการอธิบายการเป็นนักเรียนของสิ่งของที่เราสนใจอยู่ \footnote{ในเรื่องเซตจะเรียกเซตที่ระบุขอบเขตของสิ่งของที่เราสนใจว่า ``เอกภพสัมพัทธ์''} ซึ่งแน่นอนว่าเราไม่สามารถที่จะบอกค่าความจริงของตัวประโยคนี้ด้วยตัวมันเองได้ เพราะเราไม่รู้ว่าเราหมายถึง $x$ คนไหน (หรืออาจจะไม่ใช่คนตั้งแต่แรกเสียด้วยซ้ำ) เราจะเรียกประโยคประเภทนี้ว่าประโยคเปิด

แต่ก่อนจะลงลึกในเรื่องของประโยคเปิด (ซึ่งจะกล่าวถึงในบทถัดไป) จะขอกล่าวถึงแค่เฉพาะข้อความที่ระบุค่าความจริงได้ก่อน (ที่เรียกว่าประพจน์) ซึ่งในตรรกศาสตร์ เราจะนำประพจน์เหล่านี้มาเป็นตัวแทนของข้อความที่พูดกัน และนำมาเชื่อมประโยคเข้าด้วยกันด้วยตัวดำเนินการทางตรรกศาสตร์ (1) และ (2) หรือ (3) ถ้า...แล้ว... (4) ก็ต่อเมื่อ และ (5) ไม่... ซึ่งแน่นอนว่าการดูค่าความจริงของตัวเชื่อมเหล่านี้ก็เป็นการนิยามมาจากวิธีคิดของมนุษย์ที่ตกลงกันไว้และใช้กันเป็นสามัญสำนึก ดังนี้
\begin{itemize}
	\item ``และ'' จะมีบริบทการใช้งานที่เป็นการระบุการเกิดทั้งสองอย่างพร้อมกัน
	\item ``หรือ'' จะมีบริบทการใช้งานที่เป็นการระบุการเกิดอย่างน้อย 1 อย่าง (ซึ่งอาจจะต่างกับการใช้ ``หรือ'' ในภาษาไทยที่มีการใช้ในแง่คำถามให้เลือกเพียงอย่างใดอย่างหนึ่ง)
	\item ``ถ้า...แล้ว...'' จะให้ความรู้สึกของการกำหนดเงื่อนไขหรือกฏกติกาไว้ว่าเมื่อไหร่ก็ตามที่เกิดสิ่งหนึ่งขึ่นแล้วอีกสิ่งจะถูกบังคับว่าต้องเกิด มิฉะนั้นจะถือว่าเป็นการแหกกฏ
	\item ``ก็ต่อเมื่อ'' จะแทนความเป็นสิ่งเดียวกัน ใช้แทนกันได้
\end{itemize}
ซึ่งบริบทของคำเชื่อมเหล่านี้ไม่ใช่สิ่งที่แปรเปลี่ยนไปตามความเข้าใจของบุคคล แต่เป็นข้อตกลงในการตีความ
\section{ความสัมพันธ์}

\section{ฟังก์ชัน}

\section{โครงสร้างของตรรกศาสตร์อันดับหนึ่ง: บริบทและการตีความ}