\chapter{Logic, Reasoning and Proof}

หลังจากที่ผู้เขียนได้เกริ่นนำบทบาทหน้าที่ของตรรกศาสตร์ในแง่ของเครื่องมือในการสร้างประโยคและการให้เหตุผลไปในบทที่ \ref{chap:mathLang} แบบคร่าว ๆ ไปแล้ว
คราวนี้ ถึงเวลาที่ผู้อ่านจะได้ลงสู่รายละเอียดของตรรกศาสตร์กันบ้าง
ตามชื่อบท ผู้อ่านจะพบว่ามีคำ 3 อยู่ในชื่อบท ได้แก่ (1) Logic (ตรรกศาสตร์) (2) Reasoning (การให้เหตุผล) (3) Proof (การเขียนพิสูจน์) ซึ่งจะเป็น 3 ส่วนหลักที่จะอธิบายในบทนี้ ซึ่ง 3 สิ่งนี้เป็นสิ่งที่แยกขาดออกจากกันไม่ได้ เพราะเมื่อเราอยากจะเขียนพิสูจน์อะไรสักอย่าง (เหมือนเขียนรายงานเพื่อโน้มน้าวผู้อ่าน) เราก็ต้องผ่านขั้นตอนการหาเหตุผลเพื่อสรุปผลในสิ่งที่อยากพิสูจน์ ซึ่งเหตุผลที่ใช้ก็ต้องเป็นเหตุผลที่ถูกต้องตามหลักคณิตศาสตร์ และใช้ตรรกศาสตร์เป็นความรู้พื้นฐานประกอบการให้เหตุผลให้สมเหตุสมผลในเชิงคณิตศาสตร์นั่นเอง

จากที่กล่าวไป จะเห็นว่าตรรกศาสตร์เปรียบเสมือนเป็นชุดความรู้ (knowledge) เพื่อนำมาฝึกทักษะ (skill) การให้เหตุผล และเมื่อให้เหตุผลแล้ว เราต้องมีระเบียบวิธีขั้นตอน (methodology) ที่จะสามารถสื่อสารกระบวนการดังกล่าวให้ผู้อื่นเข้าใจด้วยการเขียนพิสูจน์นั่นเอง

ทั้งนี้ สำหรับผู้อ่านท่านใดที่เคยผ่านวิชาที่เกี่ยวกับการเขียนพิสูจน์มาแล้ว อาจจะข้ามบทนี้ไปก็ได้ เพราะบทนี้เป็นการปูพื้นฐานการให้เหตุผลเชิงคณิตศาสตร์สำหรับผู้ที่ยังไม่เคยเรียนคณิตศาสตร์แนวนี้มาก่อน
แต่สำหรับผู้อ่านที่ยังไม่มีประสบการณ์ในการให้เหตุผลเชิงคณิตศาสตร์ ขอให้อยู่กับบทนี้มากพอก่อนที่จะเริ่มบทถัดไป เพราะเป้าหมายหลักของหนังสือนี้คือฝึกทักษะการให้เหตุผลเชิงคณิตศาสตร์และพิสูจน์เชิงคณิตศาสตร์ ไม่ใช่หนังสือเตรียมสอบวิชาคณิตศาสตร์ และไม่ใช่หนังสือที่รวมเอาเนื้อหาของแต่ละบทมานำเสนอให้ท่องจำ (เช่นอ่านบทตรรกศาสตร์ของหนังสือเล่มนี้เข้าใจก็ไม่ได้หมายความว่าจะทำข้อสอบบทตรรกศาสตร์ของวิชา ม.4 ได้\footnote{ผู้เขียนยังทำข้อสอบเรื่องตรรกศาสตร์ในข้อสอบสอบเข้ามหาวิทยาลัยไม่ค่อยได้เช่นกันครับ}) แต่เป็นหนังสือที่จะพาผู้อ่านคิดไปด้วยกันทีละขั้นตอน ว่ากำลังจะเกิดอะไรขึ้น แล้วเกิดอะไรขึ้นมาแล้ว จะไปต่อยังไง และควรไปทางไหนต่อดี

\section{ตรรกศาสตร์คืออะไร}

ตรรกศาสตร์ ถ้าแปลตามตัวคำจะแปลว่า ศาสตร์แห่งการศึกษาตรรกะ กล่าวคือ การศึกษาเกี่ยวกับข้อความ ค่าความจริง และการให้เหตุผล



\section{การให้เหตุผลทางคณิตศาสตร์ และการพิสูจน์} \label{section:proof}





\section{การเขียนพิสูจน์}
















