\chapter{Logic, Reasoning and Proof}

หลังจากที่ผู้เขียนได้เกริ่นนำบทบาทหน้าที่ของตรรกศาสตร์ในแง่ของเครื่องมือในการสร้างประโยคและการให้เหตุผลไปในบทที่ \ref{chap:mathLang} แบบคร่าว ๆ ไปแล้ว
คราวนี้ ถึงเวลาที่ผู้อ่านจะได้ลงสู่รายละเอียดของตรรกศาสตร์กันบ้าง
ตามชื่อบท ผู้อ่านจะพบว่ามีคำ 3 อยู่ในชื่อบท ได้แก่ (1) Logic (ตรรกศาสตร์) (2) Reasoning (การให้เหตุผล) (3) Proof (การเขียนพิสูจน์) ซึ่งจะเป็น 3 ส่วนหลักที่จะอธิบายในบทนี้ ซึ่ง 3 สิ่งนี้เป็นสิ่งที่แยกขาดออกจากกันไม่ได้ เพราะเมื่อเราอยากจะเขียนพิสูจน์อะไรสักอย่าง (เหมือนเขียนรายงานเพื่อโน้มน้าวผู้อ่าน) เราก็ต้องผ่านขั้นตอนการหาเหตุผลเพื่อสรุปผลในสิ่งที่อยากพิสูจน์ ซึ่งเหตุผลที่ใช้ก็ต้องเป็นเหตุผลที่ถูกต้องตามหลักคณิตศาสตร์ และใช้ตรรกศาสตร์เป็นความรู้พื้นฐานประกอบการให้เหตุผลให้สมเหตุสมผลในเชิงคณิตศาสตร์นั่นเอง

จากที่กล่าวไป จะเห็นว่าตรรกศาสตร์เปรียบเสมือนเป็นชุดความรู้ (knowledge) เพื่อนำมาฝึกทักษะ (skill) การให้เหตุผล และเมื่อให้เหตุผลแล้ว เราต้องมีระเบียบวิธีขั้นตอน (methodology) ที่จะสามารถสื่อสารกระบวนการดังกล่าวให้ผู้อื่นเข้าใจด้วยการเขียนพิสูจน์นั่นเอง

ทั้งนี้ สำหรับผู้อ่านท่านใดที่เคยผ่านวิชาที่เกี่ยวกับการเขียนพิสูจน์มาแล้ว อาจจะข้ามบทนี้ไปก็ได้ เพราะบทนี้เป็นการปูพื้นฐานการให้เหตุผลเชิงคณิตศาสตร์สำหรับผู้ที่ยังไม่เคยเรียนคณิตศาสตร์แนวนี้มาก่อน
แต่สำหรับผู้อ่านที่ยังไม่มีประสบการณ์ในการให้เหตุผลเชิงคณิตศาสตร์ ขอให้อยู่กับบทนี้มากพอก่อนที่จะเริ่มบทถัดไป เพราะเป้าหมายหลักของหนังสือนี้คือฝึกทักษะการให้เหตุผลเชิงคณิตศาสตร์และพิสูจน์เชิงคณิตศาสตร์ ไม่ใช่หนังสือเตรียมสอบวิชาคณิตศาสตร์ และไม่ใช่หนังสือที่รวมเอาเนื้อหาของแต่ละบทมานำเสนอให้ท่องจำ (เช่นอ่านบทตรรกศาสตร์ของหนังสือเล่มนี้เข้าใจก็ไม่ได้หมายความว่าจะทำข้อสอบบทตรรกศาสตร์ของวิชา ม.4 ได้\footnote{ผู้เขียนยังทำข้อสอบเรื่องตรรกศาสตร์ในข้อสอบสอบเข้ามหาวิทยาลัยไม่ค่อยได้เช่นกันครับ}) แต่เป็นหนังสือที่จะพาผู้อ่านคิดไปด้วยกันทีละขั้นตอน ว่ากำลังจะเกิดอะไรขึ้น แล้วเกิดอะไรขึ้นมาแล้ว จะไปต่อยังไง และควรไปทางไหนต่อดี

\section{ตรรกศาสตร์คืออะไร}

ตรรกศาสตร์ ถ้าแปลตามตัวคำจะแปลว่า ศาสตร์แห่งการศึกษาตรรกะ กล่าวคือ การศึกษาเกี่ยวกับข้อความ ค่าความจริง และการให้เหตุผล



\section{การให้เหตุผลทางคณิตศาสตร์ และการพิสูจน์} \label{section:proof}

หัวใจสำคัญของคณิตศาสตร์ไม่ใช่การหาคำตอบที่ถูกต้อง แต่คือการอธิบายอย่างมีเหตุผลว่า “ทำไม” คำตอบนั้นจึงถูกต้อง \textbf{การพิสูจน์} (proof) จึงเป็นกระบวนการที่ทำให้คณิตศาสตร์แตกต่างจากศาสตร์อื่น ๆ เพราะไม่ใช่เพียงการสังเกตจากตัวอย่างหรือการคำนวณเชิงประสบการณ์ แต่คือการสร้างหลักฐานเชิงตรรกะที่ยืนยันความจริงได้ในทุกกรณี การเรียนรู้วิธีการพิสูจน์จึงเปรียบได้กับการฝึกให้คิดอย่างเป็นระบบ สื่อสารอย่างมีเหตุผล และตรวจสอบความถูกต้องของแนวคิดของตนเองได้อย่างอิสระ

ในทางปฏิบัติ “การพิสูจน์” ไม่ได้หมายถึงการเขียนข้อความยาว ๆ ด้วยถ้อยคำซับซ้อน แต่หมายถึงการรู้จักวิเคราะห์โครงสร้างของประโยคทางคณิตศาสตร์ ว่ามีตัวบ่งปริมาณ (quantifier) ประเภทใด มีรูปแบบเงื่อนไขอย่างไร และควรใช้วิธีการใดในการเชื่อมโยงสมมติฐานไปยังข้อสรุปได้อย่างถูกต้อง การพิสูจน์บางประเภทใช้ตรรกะตรงไปตรงมา (direct proof) ขณะที่บางประเภทอาศัยการแปลงรูปแบบของข้อความ (contrapositive proof) หรือแม้กระทั่งการสมมติสิ่งตรงข้ามขึ้นมาเพื่อนำไปสู่ความขัดแย้ง (proof by contradiction)  

ในหัวข้อนี้ เราจะศึกษา “วิธีการพิสูจน์” ที่เป็นรากฐานของคณิตศาสตร์เชิงตรรกะ ซึ่งปรากฏอยู่ในแทบทุกแขนงของคณิตศาสตร์และวิทยาการคอมพิวเตอร์ โดยเนื้อหาจะค่อย ๆ พาผู้อ่านเข้าใจจากระดับของประโยคพื้นฐานไปสู่ระดับที่ซับซ้อนขึ้น เริ่มจากการวิเคราะห์ข้อความที่มีตัวบ่งปริมาณสากล (for all) และข้อความเชิงเงื่อนไข (if–then) ไปจนถึงข้อความที่มีการมีอยู่ (there exists) และปิดท้ายด้วยเทคนิคการพิสูจน์ด้วยข้อขัดแย้ง (contradiction) รวมถึงส่วนเสริมเกี่ยวกับองค์ประกอบย่อยของประโยคเชิงตรรกะ เช่น “และ” (and) และ “หรือ” (or) ซึ่งมักปรากฏอยู่ในการพิสูจน์แทบทุกแบบ

จุดมุ่งหมายของบทนี้ไม่ใช่ให้ผู้อ่านจำรูปแบบการพิสูจน์ได้ แต่เพื่อให้สามารถ “เลือกใช้วิธีที่เหมาะสม” ในการอธิบายความจริงทางคณิตศาสตร์ได้ด้วยตนเอง เมื่อเข้าใจหลักการพิสูจน์แล้ว ผู้อ่านจะสามารถนำแนวคิดนี้ไปใช้ตรวจสอบความถูกต้องของนิพจน์ สมบัติของโครงสร้างทางคณิตศาสตร์ หรือแม้แต่ความถูกต้องของอัลกอริทึมที่ตนเขียนได้ในรายวิชาวิทยาการคอมพิวเตอร์ในลำดับถัดไป

\paragraph*{Key Takeaway:}
\begin{itemize}
	\item “การพิสูจน์” คือการเชื่อมสมมติฐานไปยังข้อสรุปผ่านตรรกะที่ถูกต้อง
	\item ฝึกวิเคราะห์สมมติฐานและข้อสรุปจากนิยาม
	\item บางครั้งต้องพิสูจน์กรณีทั่วไป บางครั้งต้องหาหลักฐานพยานยืนยัน
\end{itemize}
\subsection{องค์ประกอบพื้นฐานของข้อความตรรกะ}
ก่อนอื่น ควรทำความเข้าใจโครงสร้างย่อยของประโยคเชิงตรรกะที่มักปรากฏในการพิสูจน์ เช่น ข้อความที่เชื่อมด้วยคำว่า “และ” (and) หรือ “หรือ” (or) ซึ่งส่งผลโดยตรงต่อวิธีการวิเคราะห์และการแยกกรณีในการให้เหตุผล การเข้าใจการใช้คำเชื่อมเหล่านี้อย่างถูกต้องช่วยให้สามารถเขียนและอ่านพิสูจน์ได้แม่นยำขึ้น โดยเฉพาะเมื่อข้อความหนึ่งมีเงื่อนไขหลายส่วนที่ต้องพิสูจน์พร้อมกัน หรือมีหลายกรณีที่ต้องตรวจสอบแยกต่างหาก

ในแต่ละหัวข้อย่อย จะขอกล่าวถึง 2 รูปแบบที่จำเป็นคือ (1) การนำไปใช้งาน และ (2) การแสดงให้เห็นจริง

\subsubsection{ข้อความ ``และ''}
อย่างที่ทราบกันอยู่แล้วว่าตารางค่าความจริงของข้อความ ``และ'' คือ
\begin{center}
	\begin{tabular}{|c|c|c|}
		\hline
		$P$ & $Q$ & $P \land Q$ \\
		\hline
		T & T & \textbf{T} \\
		T & F & F \\
		F & T & F \\
		F & F & F \\
		\hline
	\end{tabular}
\end{center}
กล่าวคือ กรณีเดียวที่จะเป็นจริง คือกรณีที่ทั้งสองประพจน์ย่อยต้องเป็นจริง ดังนั้นการนำไปใช้งานเมื่อจะอ้างถึง (หรือสมมติมาแล้วว่าเป็นจริง) คือการนำแต่ละส่วนไปใช้งานแยกกันได้อิสระเลยโดยไม่จำเป็นต้องอ้างถึงทั้งคู่ในเวลาเดียวกัน

ในขณะที่การจะแสดงให้เห็นจริงของประโยค ``และ'' คือการพิสูจน์ประพจน์ย่อยทั้งสองให้เห็นจริงทีละตัว

\subsubsection{ข้อความ ``หรือ''}

\subsubsection{การปฏิเสธ}

\subsection{การพิสูจน์ข้อความที่มีตัวบ่งปริมาณ \textit{for all}}
ข้อความที่มีตัวบ่งปริมาณแบบ “for all” หรือ $\forall$ มักเป็นข้อความที่กล่าวถึงความจริงสากล เช่น “สำหรับทุกจำนวนเต็ม $n$ จะเป็นจริงว่า …” การพิสูจน์ข้อความลักษณะนี้มักเป็นการเลือกวัตถุทั่วไปหนึ่งตัว (arbitrary element) แล้วแสดงว่าข้อความนั้นเป็นจริงในกรณีทั่วไป ซึ่งถือเป็นรากฐานของการพิสูจน์แบบตรง (direct proof) นักศึกษาจึงต้องเข้าใจทั้งนิยามของคำสำคัญที่ปรากฏในข้อความ และวิธีเชื่อมโยงสมมติฐานไปยังข้อสรุปโดยไม่อาศัยการคาดเดาจากตัวอย่างเฉพาะ

\subsection{การพิสูจน์ข้อความเงื่อนไขผลสรุป}
ข้อความเชิงเงื่อนไข (conditional statement) มีรูปทั่วไปว่า “ถ้า $P$ แล้ว $Q$” ซึ่งในทางตรรกะหมายถึงว่า เมื่อสมมติว่าข้อความ $P$ เป็นจริง ข้อความ $Q$ จะต้องเป็นจริงด้วย การพิสูจน์ข้อความลักษณะนี้จึงเป็นการแสดงให้เห็นว่าความจริงของ $P$ เพียงพอที่จะนำไปสู่ความจริงของ $Q$ ได้ วิธีการที่ใช้มีหลายแบบ เช่น การพิสูจน์ตรง (direct proof), การพิสูจน์โดยรูปกลับ (contrapositive proof) หรือการพิสูจน์โดยข้อขัดแย้ง (contradiction proof) ซึ่งแต่ละวิธีจะมีแนวคิดและโครงสร้างการให้เหตุผลต่างกันไป

\subsubsection{Direct Proof}
การพิสูจน์แบบตรง (Direct Proof) เป็นวิธีพื้นฐานที่สุดของการพิสูจน์ข้อความเชิงเงื่อนไข โดยเริ่มจากสมมติว่าข้อความ $P$ เป็นจริง แล้วใช้ข้อเท็จจริง นิยาม หรือทฤษฎีที่ทราบอยู่เดิม เพื่ออนุมานไปทีละขั้นจนได้ข้อสรุป $Q$ วิธีนี้เป็นรูปแบบของการให้เหตุผลเชิงตรรกะที่ตรงไปตรงมา และช่วยฝึกการอธิบายอย่างมีลำดับชัดเจน เช่น การพิสูจน์ว่า “ถ้า $n$ เป็นจำนวนคู่แล้ว $n^2$ เป็นจำนวนคู่” ซึ่งสามารถทำได้โดยใช้เพียงนิยามของจำนวนคู่และการคำนวณทางพีชคณิตอย่างง่าย

\subsubsection{Contrapositive Proof}
การพิสูจน์โดยรูปกลับ (Contrapositive Proof) อาศัยสมบัติที่ว่า ข้อความ “ถ้า $P$ แล้ว $Q$” สมมูลกับ “ถ้าไม่ใช่ $Q$ แล้วไม่ใช่ $P$” (If $P \Rightarrow Q$, then $\neg Q \Rightarrow \neg P$) วิธีนี้มักใช้เมื่อการพิสูจน์ตรงทำได้ยาก แต่การพิสูจน์รูปกลับง่ายกว่า เพราะสามารถใช้การปฏิเสธของข้อสรุปมาช่วยวิเคราะห์เงื่อนไขที่จำเป็นได้ วิธีนี้พบได้บ่อยในทฤษฎีจำนวนและโครงสร้างข้อมูล เช่น การพิสูจน์ว่าถ้า $n^2$ หารด้วย 3 ลงตัวแล้ว $n$ ต้องหารด้วย 3 ลงตัวด้วย

\subsubsection{Proof by Contradiction}
การพิสูจน์โดยข้อขัดแย้ง (Proof by Contradiction) เป็นเทคนิคที่ทรงพลังและใช้ได้กับข้อความทุกชนิด หลักการคือสมมติว่าข้อความที่ต้องการพิสูจน์เป็นเท็จ แล้วใช้เหตุผลเชิงตรรกะต่อเนื่องจนได้ผลลัพธ์ที่ขัดแย้งกับข้อเท็จจริงที่ทราบอยู่ หรือแม้แต่ขัดแย้งกับสมมติฐานของตนเอง เมื่อเกิดความขัดแย้งขึ้น แสดงว่าสมมติฐานเดิมต้องผิด ดังนั้นข้อความที่ต้องการพิสูจน์จึงเป็นจริง วิธีนี้มักใช้เมื่อการพิสูจน์โดยตรงซับซ้อนเกินไป ตัวอย่างที่คลาสสิกคือการพิสูจน์ว่า $\sqrt{2}$ เป็นจำนวนอตรรกยะ

\subsection{การพิสูจน์ข้อความที่มีตัวบ่งปริมาณ \textit{there exist}}
ข้อความที่มีตัวบ่งปริมาณแบบ “there exists” หรือ $\exists$ เป็นข้อความที่ยืนยันการมีอยู่ของวัตถุใดวัตถุหนึ่งซึ่งทำให้เงื่อนไขเป็นจริง การพิสูจน์ข้อความประเภทนี้จึงมีสองแนวทางหลัก ได้แก่ (1) การพิสูจน์แบบสร้างสรรค์ (constructive proof) ซึ่งแสดงวัตถุที่ต้องการโดยตรง และ (2) การพิสูจน์แบบไม่สร้างสรรค์ (non-constructive proof) ซึ่งพิสูจน์ว่าการไม่มีวัตถุดังกล่าวจะนำไปสู่ความขัดแย้ง แม้ไม่สามารถระบุวัตถุนั้นได้อย่างชัดเจนก็ตาม การแยกแยะระหว่างสองแนวทางนี้ช่วยให้เข้าใจมิติของ “การมีอยู่” ในคณิตศาสตร์ได้ลึกซึ้งยิ่งขึ้น

\subsection{การพิสูจน์ด้วยข้อขัดแย้ง (ทั่วไป)}
แม้ว่าการพิสูจน์ด้วยข้อขัดแย้ง (Proof by Contradiction) จะถูกใช้เป็นเทคนิคย่อยในหลายบริบท แต่ก็สามารถมองว่าเป็นวิธีการทั่วไปที่ใช้ตรวจสอบความสมเหตุสมผลของข้ออ้างใด ๆ ได้เช่นกัน หลักคิดสำคัญคือ “สิ่งที่ขัดแย้งกับความจริงย่อมเป็นเท็จ” ดังนั้น หากเราสมมติว่าข้อความที่ต้องการพิสูจน์เป็นเท็จ แล้วการให้เหตุผลของเรานำไปสู่ความขัดแย้ง เช่น ได้ผลลัพธ์ที่ขัดกับนิยาม หรือขัดกับหลักพื้นฐานทางคณิตศาสตร์ ก็สามารถสรุปได้ว่าข้อความนั้นต้องเป็นจริง วิธีนี้มีพลังมากในกรณีที่ข้อความเกี่ยวข้องกับการมีอยู่ (existence) หรือความเป็นไปไม่ได้ (impossibility)



\section{ข้อควรคำนึงในการเขียนพิสูจน์}

\subsection*{การเริ่มต้นโดยไม่ระบุสิ่งที่ต้องพิสูจน์ให้ชัดเจน}
ในบทพิสูจน์ที่ดี ผู้อ่านควรรู้ทันทีว่าผู้เขียนต้องการพิสูจน์อะไร เช่น ต้องการแสดงว่า “ทุกจำนวนคู่มีรูป $2k$” หรือ “ไม่มีจำนวนเฉพาะคู่ใดนอกจาก 2”  
\begin{exam}
	\textbf{ผิด:} “สมมติว่า $n$ เป็นจำนวนคู่ ดังนั้น $n = 2k$.” (ไม่มีการระบุว่าต้องการพิสูจน์อะไร)\\
	\textbf{ถูก:} “เราต้องการพิสูจน์ว่า ผลคูณของจำนวนคู่สองจำนวนเป็นจำนวนคู่ เริ่มจากให้ $n=2a$ และ $m=2b$ จากนั้น $nm = 4ab = 2(2ab)$ ซึ่งเป็นจำนวนคู่.”
\end{exam}

\subsection*{การใช้ตัวแปรโดยไม่ระบุขอบเขต}
หลายครั้งนักศึกษาจะเขียนตัวแปรขึ้นมาโดยไม่บอกว่ามันเป็นจำนวนเต็ม จำนวนจริง หรือจำนวนบวก ซึ่งทำให้ความหมายของพิสูจน์ไม่ชัดเจน  
\begin{exam}
	\textbf{ผิด:} “ให้ $n^2$ เป็นจำนวนคู่ ดังนั้น $n$ เป็นจำนวนคู่.”\\
	\textbf{ถูก:} “ให้ $n \in \mathbb{Z}$ และสมมติว่า $n^2$ เป็นจำนวนคู่ เราต้องการพิสูจน์ว่า $n$ เป็นจำนวนคู่.”
\end{exam}

\subsection*{การใช้ตัวอย่างแทนการพิสูจน์}
การยกตัวอย่างเพียงบางกรณีไม่ถือว่าเป็นการพิสูจน์ เพราะคณิตศาสตร์ต้องการความจริงที่ใช้ได้กับทุกกรณี  
\begin{exam}
	\textbf{ผิด:} “เพราะ $2+4=6$ และ $4+6=10$ ซึ่งเป็นจำนวนคู่ ดังนั้นผลบวกของจำนวนคู่สองจำนวนเป็นจำนวนคู่.”\\
	\textbf{ถูก:} “ให้ $a=2m$ และ $b=2n$ จากนั้น $a+b=2(m+n)$ ซึ่งหารด้วย 2 ลงตัว ดังนั้นผลบวกของจำนวนคู่สองจำนวนเป็นจำนวนคู่เสมอ.”
\end{exam}

\subsection*{การเริ่มจากสิ่งที่ต้องการพิสูจน์ (ย้อนเหตุผล)}
การพิสูจน์ต้องเริ่มจากสมมติฐาน ไม่ใช่จากสิ่งที่ต้องการพิสูจน์ เพราะจะกลายเป็นการสมมติว่าคำตอบถูกอยู่แล้ว  
\begin{exam}
	\textbf{ผิด:} “สมมติว่า $n^2$ เป็นคู่ ดังนั้น $n$ ต้องเป็นคู่ เพราะเราอยากให้เป็นอย่างนั้น.”\\
	\textbf{ถูก:} “สมมติว่า $n$ เป็นคี่ เขียนได้เป็น $n=2k+1$ จากนั้น $n^2=4k^2+4k+1$ ซึ่งเป็นคี่ ดังนั้นหาก $n^2$ เป็นคู่ $n$ ต้องไม่เป็นคี่.”
\end{exam}

\subsection*{การละขั้นตอนสำคัญโดยไม่อธิบาย}
การพิสูจน์ที่ดีต้องให้เหตุผลต่อเนื่อง ไม่กระโดดจากสมมติฐานไปสู่ข้อสรุปทันที  
\begin{exam}
	\textbf{ผิด:} “เพราะ $a|b$ ดังนั้น $a|bc$.”\\
	\textbf{ถูก:} “เพราะ $a|b$ จะมีจำนวนเต็ม $k$ ที่ทำให้ $b=ak$ ดังนั้น $bc = a(kc)$ ซึ่งแสดงว่า $a|bc$.”
\end{exam}

\subsection*{การสลับทิศทางของสมมูล}
นักศึกษามักสับสนระหว่างการพิสูจน์แบบ “ถ้า–แล้ว” ($\Rightarrow$) กับแบบ “ถ้าและเฉพาะถ้า” ($\Leftrightarrow$)  
\begin{exam}
	\textbf{ผิด:} “เพราะ $n$ เป็นคู่ ดังนั้น $n^2$ เป็นคู่ ดังนั้นกลับกัน $n^2$ เป็นคู่ $\Rightarrow n$ เป็นคู่.”\\
	\textbf{ถูก:} “เราได้พิสูจน์แล้วว่า $n$ เป็นคู่ $\Rightarrow n^2$ เป็นคู่ ส่วนกลับพิสูจน์ได้โดย contrapositive ว่า ถ้า $n^2$ เป็นคู่ $\Rightarrow n$ เป็นคู่.”
\end{exam}

\subsection*{การใช้สัญลักษณ์โดยไม่สอดคล้องกับภาษา}
ข้อความคณิตศาสตร์ที่ดีควรเชื่อมโยงระหว่างภาษากับสัญลักษณ์อย่างเหมาะสม หากใช้แต่สัญลักษณ์หรือแต่ภาษาอย่างเดียวจะอ่านยากและเสี่ยงต่อความคลาดเคลื่อน  
\begin{exam}
	\textbf{ผิด:} “เพราะเท่ากับเลยหารได้ลงตัว.” (ไม่ระบุว่าสิ่งใดเท่ากับสิ่งใด)\\
	\textbf{ถูก:} “จาก $n = 2k$ จะได้ว่า $n$ หารด้วย 2 ลงตัว เนื่องจากสามารถเขียนเป็นผลคูณของ 2 กับจำนวนเต็ม $k$.”
\end{exam}

\subsection*{การปฏิเสธข้อความผิดรูป}
เมื่อปฏิเสธข้อความที่มีตัวบ่งปริมาณ มักผิดเพราะลืมสลับชนิดของตัวบ่งปริมาณ  
\begin{exam}
	\textbf{ผิด:} ปฏิเสธ “ทุกคนรักคณิตศาสตร์” ด้วย “ไม่มีใครรักคณิตศาสตร์.”\\
	\textbf{ถูก:} ปฏิเสธคือ “มีบางคนที่ไม่รักคณิตศาสตร์.” ซึ่งสอดคล้องกับสมการตรรกะ $\neg(\forall x, P(x)) \equiv \exists x, \neg P(x)$
\end{exam}

\subsection*{การไม่แยกกรณีเมื่อต้อง}
บางข้อความต้องแยกกรณีเพื่อให้ครบถ้วน โดยเฉพาะเมื่อค่าตัวแปรมีหลายลักษณะ  
\begin{exam}
	\textbf{ผิด:} “ผลคูณของจำนวนเต็มสองจำนวนเป็นบวกเสมอ.”\\
	\textbf{ถูก:} “ถ้าทั้งคู่เป็นบวก ผลคูณเป็นบวก และถ้าทั้งคู่เป็นลบ ผลคูณก็เป็นบวก แต่ถ้าเครื่องหมายต่างกัน ผลคูณจะเป็นลบ.”
\end{exam}

\subsection*{การไม่ตรวจสอบเงื่อนไขที่ซ่อนอยู่ในนิยาม}
นิยามบางอย่างมีข้อจำกัดที่มักถูกมองข้าม เช่น ห้ามหารด้วยศูนย์ หรือฟังก์ชันต้องมีโดเมนที่ชัดเจน  
\begin{exam}
	\textbf{ผิด:} “เพราะ $\frac{a}{b}=c$ ดังนั้น $a=bc$.” (ไม่ระบุว่า $b \neq 0$)\\
	\textbf{ถูก:} “สมมติว่า $b \neq 0$ และ $\frac{a}{b}=c$ จะได้ว่า $a=bc$.”
\end{exam}

การเรียนรู้วิธีการพิสูจน์ไม่ได้เป็นเพียงการจดจำรูปแบบของการให้เหตุผล แต่คือการฝึกฝนวิธีคิดที่เป็นระบบ มีเหตุผล และสามารถอธิบายความจริงได้อย่างมั่นใจ ในตอนต้นของบทนี้ เราได้เริ่มจากโครงสร้างพื้นฐานของประโยคตรรกะ เช่น การเชื่อมด้วย “และ” หรือ “หรือ” จากนั้นจึงขยายไปสู่ข้อความที่มีตัวบ่งปริมาณ “สำหรับทุกค่า” และ “มีอยู่ค่าใดค่าหนึ่ง” ก่อนจะเรียนรู้วิธีการพิสูจน์รูปแบบต่าง ๆ ไม่ว่าจะเป็นการพิสูจน์ตรง การพิสูจน์โดยรูปกลับ หรือการพิสูจน์ด้วยข้อขัดแย้ง ซึ่งล้วนสะท้อนแนวทางของการให้เหตุผลที่มีตรรกะเป็นแกนกลาง

อย่างไรก็ตาม สิ่งที่สำคัญไม่แพ้เทคนิคคือ “การสื่อสารทางคณิตศาสตร์” — การเขียนให้ผู้อ่านเข้าใจว่าคุณกำลังคิดอะไร อยู่บนสมมติฐานใด และสรุปผลอย่างไร หลายครั้งที่พิสูจน์ผิดไม่ได้เพราะตรรกะผิด แต่เพราะไม่ระบุขอบเขตของตัวแปร ไม่แยกกรณี หรือเขียนโดยไม่ระบุว่ากำลังพิสูจน์อะไร ดังนั้น การเขียนพิสูจน์ที่ดีจึงต้องทั้งถูกต้องในเชิงตรรกะและชัดเจนในเชิงภาษา

ท้ายที่สุด การฝึกเขียนพิสูจน์คือการฝึกตั้งคำถามกับสิ่งที่เราคิดว่าจริง — “เรารู้ได้อย่างไรว่ามันจริง?” เมื่อผู้เรียนเริ่มตั้งคำถามเช่นนี้ นั่นคือจุดเริ่มต้นของการคิดเชิงคณิตศาสตร์อย่างแท้จริง ซึ่งจะกลายเป็นรากฐานของการทำงานด้านวิทยาการคอมพิวเตอร์ การออกแบบอัลกอริทึม และการให้เหตุผลในชีวิตจริงต่อไป















