\chapter*{Prologue}

\section*{819605 Discrete Mathematics (2/2568)}

สำหรับการสอนรายวิชา 819605 Discrete Mathematics ของหลักสูตรวิศวกรรมคอมพิวเตอร์ออนไลน์ มหาวิทยาลัยเอเชียอาคเนย์ ในภาคการศึกษาปลาย ประจำปีการศึกษา 2568 นี้ ถือเป็นครั้งแรกที่ผู้เขียนได้สอนวิชา Discrete Mathematics อย่างเป็นทางการในฐานะอาจารย์มหาวิทยาลัย หลังจากที่เคยสอนวิชานี้ในรูปแบบติวเตอร์มานับครั้งไม่ถ้วน การสอนในฐานะติวเตอร์นั้นมักเป็นเพียงการถ่ายทอดเนื้อหาตามที่สถาบันติวได้จัดเตรียมไว้ให้ ซึ่งหลายครั้งก็ทำให้ผู้สอนรู้สึกสงสัยว่าทำไมเนื้อหาบางส่วนถูกจัดมาโดยข้ามหัวข้อสำคัญ หรือไม่เรียงตามตรรกะของการเรียนรู้ที่ควรจะเป็น  

ในฐานะอาจารย์ ผู้เขียนจึงมีโอกาสได้ออกแบบลำดับการสอนใหม่ทั้งหมด ตั้งแต่แนวคิด วิธีอธิบาย ไปจนถึงแบบฝึกหัด โดยพยายามรักษาสมดุลระหว่าง “ความเป็นคณิตศาสตร์” ที่เข้มงวด กับ “ความเป็นวิทยาการคอมพิวเตอร์” ที่มุ่งใช้งานจริง โดยเฉพาะในส่วนของการสอนเรื่องการพิสูจน์ (Method of Proof) ซึ่งแม้จะเป็นหัวใจของวิชานี้ แต่ก็ไม่จำเป็นต้องแยกสอนเป็นบท ๆ แบบที่นักศึกษาคณะคณิตศาสตร์เรียนกัน หากแต่สามารถบูรณาการเข้าไปในหัวข้อต่าง ๆ เพื่อให้ผู้เรียนเห็นการประยุกต์ของแต่ละวิธีพิสูจน์อย่างเป็นธรรมชาติและต่อเนื่อง  

ดังนั้น หนังสือเล่มนี้จึงจัดทำขึ้นเพื่อใช้ประกอบการสอนและเป็นคู่มือให้กับผู้เรียน โดยเนื้อหาการสอนจริงในห้องเรียนอาจไม่ได้เรียงตรงตามหนังสือทั้งหมด แต่จะมีโครงสร้างหลักตามแผนการสอน 16 สัปดาห์ดังต่อไปนี้ ซึ่งระบุหัวข้อการสอน วิธีพิสูจน์ที่เรียนรู้ และบทที่สอดคล้องกับหนังสือ

\scalebox{0.7}{\begin{tabular}{|c|l|l|c|}
	\hline
	\textbf{Week} & \textbf{Main Topic} & \textbf{Method of Proof} & \textbf{Book Chapter} \\
	\hline
	1 & Mathematical Language \& Sets; Recursive Definitions & Direct Proof, Counterexample & Ch.1 \\
	2 & Propositional Logic \& Inference Rules & Direct, Contrapositive, Contradiction & Ch.2 \\
	3 & Predicate Logic \& Quantifiers & Contradiction, Countermodel & Ch.2 \\
	4 & Induction I: Basic Induction on $\mathbb{N}$ & Weak Induction, Base \& Step Structure & Ch.3 \\
	5 & Induction II: Strong \& Structural Induction & Structural Induction, Minimal Counterexample & Ch.3 \\
	6 & Functions \& Relations & Proof from Definitions, Counterexample & Ch.4 \\
	7 & Equivalence Relations \& Partial Orders & Proof from Properties, Diagram Reasoning & Ch.4 \\
	8 & \textbf{Midterm Exam} & --- & --- \\
	9 & Counting I: Basic Rules \& Bijections & Combinatorial Proof, Bijection & Ch.5 \\
	10 & Counting II: Inclusion–Exclusion \& Pigeonhole & Inductive Combinatorial Proof & Ch.5 \\
	11 & Recursive Problem Solving & Inductive Reasoning in Algorithms & Ch.6 \\
	12 & Recurrences I: Linear Homogeneous & Inductive Verification of Closed Forms & Ch.6 \\
	13 & Recurrences II: Non-homogeneous \& OGFs & Induction + Generating Function Sketch & Ch.6 \\
	14 & Graphs I: Fundamentals \& Connectivity & Parity / Invariant Proofs & Ch.7 \\
	15 & Trees \& Spanning Trees & Induction on Graph Size, Cut/Cycle Property & Ch.7 \\
	16 & \textbf{Final Exam} & --- & --- \\
	\hline
\end{tabular}}
\vspace{1em}

วิชานี้จึงมิได้มีเป้าหมายเพียงให้ผู้เรียนสามารถคำนวณผลลัพธ์ได้ถูกต้องตามสูตรหรืออัลกอริทึมเท่านั้นหากแต่ต้องการให้ผู้เรียนเข้าใจ “เหตุผลเบื้องหลัง” ว่าทำไมวิธีการนั้นจึงถูกต้อง และเงื่อนไขใดที่ทำให้คำตอบสมบูรณ์ การเรียนรู้ในรายวิชา Discrete Mathematics จึงเปรียบได้กับการฝึกคิดอย่างเป็นระบบ ฝึกตั้งคำถาม ฝึกแยกแยะระหว่างสิ่งที่ “เชื่อได้” กับสิ่งที่ “ควรพิสูจน์ให้ได้” การให้เหตุผล (reasoning) จึงเป็นหัวใจสำคัญเหนือการคำนวณ เพราะคณิตศาสตร์ที่แท้จริงไม่ใช่เรื่องของตัวเลข แต่คือศิลปะแห่งความเข้าใจ — ความสามารถในการอธิบาย “ว่าทำไม” สิ่งหนึ่งจึงจริง ไม่ใช่เพียง “ว่าอะไร” เป็นคำตอบที่ถูกต้อง

ด้วยเหตุนี้ ผู้เขียนหวังว่าผู้เรียนจะไม่มองว่าวิชานี้เป็นวิชาแก้โจทย์คณิตศาสตร์เท่านั้น แต่จะเป็นวิชาที่เป็นเหมือนสนามฝึกคิด ฝึกให้เหตุผล และฝึกมองเห็นความงามของตรรกะที่อยู่เบื้องหลังทุกหลักฐาน ทุกสูตร และทุกอัลกอริทึมที่เราจะได้พบตลอดทั้งภาคการศึกษา
