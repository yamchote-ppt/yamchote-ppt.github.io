\chapter{Categories}
\section{นิยามและตัวอย่าง}
\subsection{จากสิ่งที่อยากได้ก่อนนิยามทางคณิตศาสตร์}

ถ้าผู้ที่กำลังอ่านหนังสือเล่มนี้ไม่ใช่นักคณิตศาสตร์ ก็คงเป็นการดีที่จะเริ่มมองจากสิ่งที่ใกล้ตัวก่อนหรือคุ้นเคยก่อนแล้วค่อยไปสู่นิยามทางคณิตศาสตร์ที่มีความเป็นนามธรรมมากขึ้น แต่ถ้าท่านคุ้นเคยกับการทำความเข้าใจจากสิ่งที่เป็นนามธรรมแบบที่นักคณิตศาสตร์ส่วนใหญ่เป็นอยู่แล้วนั้น ท่านอาจจะข้ามไปหัวข้อ \ref{def_cat} เลยก็ได้ครับ

ถ้าจะถามว่า category theory คืออะไรนั้น ก็ตรงตัวตามคำที่ใช้ครับว่ามันคือทฤษฎีที่ศึกษาเกี่ยวกับ \textbf{หมวดหมู่} (categories) \index{category}\index{หมวดหมู่} ของสิ่งของ ทว่าก่อนที่จะกล่าวถึงนิยามทางคณิตศาสตร์ที่อาจจะดูอ่านยากและทำความเข้าใจยาก(สำหรับหลาย ๆ คน) เรามาเริ่มจากมุมมองในแง่ว่าเราอยากได้อะไรบ้างในการออกแบบสิ่งใดสิ่งหนึ่งกันครับ

\subsubsection{เมื่อของแต่ละสิ่งมีหมวดหมู่ของตัวมันเอง}
สำหรับใครที่มีประสบการณ์ในการเขียนโปรแกรมโดยใช้ภาษาที่ต้องประกาศประเภทของตัวแปรก่อน (เช่น Java, C) อาจจะคุ้นเคยกับแนวความคิดที่ว่าของแต่ละสิ่ง (ตัวแปร) ก็มีประเภทของมัน ถ้าประกาศไว้แล้วว่าจะให้ตัวแปร $x$ เป็น integer แล้วเราจะสามารถนำมันไปใช้ได้แค่กับตำแหน่งที่ยอมให้ integer เข้าไปได้เท่านั้น รวมไปถึงการจะนำมันเป็นเก็บค่าก็จะเก็บได้กับเฉพาะค่าที่เป็น integer จริง ๆ ทั้งนี้ก็เพื่อ.....(หาแหล่งอ้างอิงเพิ่มเกี่ยวกับประโยชน์ของการประกาศประเภทตัวแปรไว้ก่อน).....

ในทำนองเดียวกับการประกาศประเภทของตัวแปรในโปรแกรมมิ่ง เราก็สามารถมองให้ของแต่ละสิ่งที่จะเกิดขึ้นมีหมวดหมู่ของตัวมันเองได้เช่นกัน เช่นมองว่านาย A เป็นนักศึกษาเป็นต้น เป็นเหมือนข้อตกลงกันแล้วว่าสิ่งที่นาย A จะทำได้หรือไม่ได้นั้น ขึ้นกับว่าหมวดหมู่นักเรียนทำได้หรือไม่ได้ และเช่นเดียวกันกับคนอื่น ๆ ที่จัดเป็นหมวดหมู่นักเรียนก็จะสามารถทำได้เหมือนนาย A กล่าวคือ ในแนวคิดของการออกแบบที่กำลังจะศึกษานั้น เราจะกล่าวกันด้วยแนวความคิดที่ว่าของสิ่งนั้นเป็นของหมวดหมู่อะไรเพื่อที่จะรู้ขอบเขตการใช้งานหรือการดำเนินการที่เกี่ยวข้องต่อไป
\subsubsection{การเชื่อมโยง}
\lipsum[1]
\subsubsection{ออกจากการลงรายละเอียดมามองภาพรวมก่อน}
\lipsum[1]
\subsection{นิยามทางคณิตศาสตร์ของ Categories} \label{def_cat}
~