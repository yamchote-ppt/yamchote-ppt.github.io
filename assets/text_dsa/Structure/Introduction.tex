% !TEX root = ../BookTemplate.tex
%%%%%%%%%%%%%%%%%%%%%%%%%%%%%%%%%%%%%%%%%%%%%%%%%%%%%%%%%%%%%%%%%%%%%%%%%%%%%%%%%%
\chapter*{}\addcontentsline{toc}{part}{\IntroductionName}
\vspace{-1.5cm}
    \begin{minipage}[r]{.95\textwidth}\raggedleft
    \HUGE\bfseries\IntroductionName
    \end{minipage}

\noindent
                    %%%%%%%%%%%%%%%%%%%%%%%%%%%%                  (EDIT BELOW)
% ก่อนจะขึ้นเนื้อหาจริง ๆ มีอีกสิ่งที่อยากจะเน้นย้ำคือวิชานี้ไม่ใช่วิชาคณิตศาสตร์ แต่เป็นวิชาที่มีคณิตศาสตร์เป็นเครื่องมือเพื่อแก้ปัญหา โดยเฉพาะปัญหาทางธุรกิจ ดังนั้นวิธีการเรียนอาจจะแตกต่างจากการเรียนวิชาคณิตศาสตร์อย่างเดียวที่เน้นไปที่การทำความเข้าใจเครื่องมือและเข้าใจที่มาว่าทำไมถึงแก้ปัญหาได้ แต่อาจจะไม่ได้แตะการนำไปแก้ปัญหาโลกจริง แต่ไม่ใช่ว่ากลุ่มนักคณิตศาสตร์จะไม่ได้เรียนการเอาไปแก้ปัญหานะครับ แต่เพียงแค่พวกเขาเหล่านั้นเรียนการแก้ปัญหาในรูปแบบที่เรียกว่าการทำให้เป็นรูปแบบนามธรรม (abstractization) ซึ่งเป็นอีกมุมมองของการทำ problem-solving แต่สำหรับวิชาทางนี้นั้น สิ่งที่เราให้ความสำคัญคือการเปลี่ยนปัญหาโลกจริงหรือปัญหาทางธุรกิจเป็นปัญหาทางคณิตศาสตร์ และสร้างตัวแบบหรือแบบจำลองเพื่อที่จะได้นำเครื่องมือมาใช้ได้ถูกต้อง และเมื่อแก้ปัญหาได้ก็ต้องตีความผลลัพธ์ได้ ดังนั้นทำให้การเรียนวิชานี้เน้นไปเรื่องต่าง ๆ ดังนี้
% \begin{enumerate}
%     \item การแปลงปัญหาโลกจริงให้อยู่ในรูปแบบคณิตศาสตร์ (มองคณิตศาสตร์เป็นภาษา)
%     \item สร้างตัวแบบ/แบบจำลองของปัญหานั้นขึ้นมาได้ (ระบุ framework ของปัญหา ระบุองค์ประกอบของปัญหานั้นได้ เช่นอะไรคือสิ่งที่ต้องการ อะไรคือสิ่งที่เป็นเงื่อนไขที่โจทย์กำหนด)
%     \item แก้ปัญหานั้นด้วยเครื่องมือที่มี โดยรู้ข้อจำกัดขอองเครื่องมือต่าง ๆ ที่ใช้ (ในหนังสือเล่มนี้จะเน้นที่ส่วนนี้ โดยให้เห็นแง่มุมของการได้มาซึ่งวิธีการแก้ปัญหา จะไม่ได้เน้นว่าแก้อย่างไรตั้งแต่แรก)
%     \item ตีความผลลัพธ์เชิงความหมายทางธุรกิจได้
% \end{enumerate}

%\section{วัดพื้นฐานคณิตศาสตร์}
%% ---------- ตัวอย่างข้อ 1 ----------
%\begin{enumerate}
%% ---------- 1–5  Arithmetic & Percent ----------
%\item ร้านค้าขายสินค้า A ราคา 280 บาท  
%      ถ้าลดราคาอีก 15\% จะเหลือขาย  
%      \[
%        280 \times \blank{1.5cm}= \blank{2cm}\ \text{บาท}
%      \]
%
%\item ต้นทุนสินค้า 150 บาท ต้องการกำไรขั้นต้น 30\%  
%      ควรตั้งราคาขาย  
%      \[
%        150 \times (1+\blank{1cm}) = \blank{2cm}\ \text{บาท}
%      \]
%
%\item เงิน 80\,000 บาท ดอกเบี้ยทบต้น 5\% ต่อปี 3 ปี  
%      มูลค่าในอนาคต \[FV = 80{,}000 (1+\blank{0.8cm})^{\blank{0.8cm}} = \blank{2.5cm} \text{ บาท}\] 
%
%\item ลดอัตราของเสียจาก 8\% เหลือ 5\%  
%      การลดลงคิดเป็น  
%       \(\displaystyle
%        \frac{8-5}{\blank{1cm}} = \blank{1.8cm}\ (\%)
%      \)
%
%\item หลังหักภาษี 20\% ของกำไรก่อนภาษีแล้วเหลือกำไรสุทธิ 2.4 ล้านบาท        จะได้ว่ากำไรก่อนภาษี คิดเป็น \[ \dfrac{2.4}{\blank{1cm}} = \blank{2cm} \text{ ล้านบาท}\]
%
%% ---------- 6–10  Algebra ----------
%\newpage
%\item แก้ระบบ  
%      \(
%        3x+2y=18\ -- (1),\; 5x-4y=2\ -- (2)
%      \)  
%      ด้วยวิธีกำจัด \(y\): 
%    \begin{enumerate}
%        \item ทำสัมประสิทธิ์ของ $y$ ของทั้งสองสมการให้เป็น 4 เท่ากันด้วยการคูณสมการที่ (1) ด้วย \blank{1.2cm} จะได้สมการออกมาเป็น
%        \[
%        \blank{1.2cm}x + 4y = \blank{1.2cm}\ --(3)
%        \]
%        \item เนื่องจาก $4y + (-4y) = 0$ จึงนำสมการที่ (2) และ (3) มาบวกกันเพื่อให้พจน์ $y$ ของทั้งสองสมการหักล้างกัน จึงได้ว่า
%        \begin{align*}
%            (\blank{1.2cm}x + 4y) + (5x-4y) &= \blank{1.2cm} + \blank{1.2cm}\\
%            \blank{1.2cm}x &= \blank{1.2cm}\\
%            x &= \frac{\blank{1.2cm}}{\blank{1.2cm}} = \blank{1.2cm}
%        \end{align*}
%        \item นำค่า $x$ ที่ได้มาหาค่า $y$ ด้วยการแทนค่าลงไปในสมการ ซึ่งในที่นี้ ขอใช้สมการที่ (1) จะได้ว่า
%        \begin{align*}
%            3x + 2\blank{1.2cm} &= 18\\
%            3x + \blank{1.2cm} &= 18\\
%            3x &= \blank{1.2cm}\\
%            x &= \blank{1.2cm}
%        \end{align*}
%
%
%    \end{enumerate}
%
%\item ต้นทุนรวม \(C=12q+3000\), รายได้ \(R=20q\)  
%      จุดคุ้มทุนหมายถึงจุดที่ต้นทุนรวมและรายได้มีค่าเท่ากันพอดี กล่าวคือ \(C = R\) หรือก็คือ \(\blank{1.2cm}q+\blank{1.2cm} = \blank{1.2cm}q \Rightarrow 3000 = \blank{1cm}\,q\)  
%      ดังนั้น \(q=\blank{1.5cm}\) หน่วย
%
%
%% ---------- 14–17  Probability ----------
%\item ถ้าความน่าจะเป็นที่จะมีลูกค้าเข้ามาในร้านไม่เกิน 6 คนต่อชั่วโมงมีค่าเท่ากับ 0.8 แล้วความน่าจะเป็นที่จะมีลูกค้าเข้ามาในร้านตั้งแต่ 7 คนขึ้นไปจะเท่ากับ \(1-\blank{1.5cm} = \blank{1.5cm}\)
%
%% ---------- 18–20  Descriptive Stats ----------
%\item ข้อมูล 12,15,17,20,26,30  
%      มัธยฐาน \(= \dfrac{\blank{1cm}+\blank{1cm}}{2} = \blank{1cm}\)
%
%
%% ---------- 21–23  Matrices ----------
%
%% ---------- 24–25  Calculus ----------
%% \item \(S(t)=50e^{0.04t}\)  
%%       Growth rate \(S'(t)=0.04\times50e^{0.04t}\)  
%%       ต่างกันระหว่าง \(t=3\) และ 6 เดือน = \(\blank{3cm}\)
%
%% \item \(AC(q)=\dfrac{15}{q}+2q\)  
%%       หา \(AC'(q)= -\dfrac{15}{q^2}+2\)  ให้เท่ากับ 0 ⇒ \(q=\sqrt{\blank{1cm}/2}\approx\blank{1.5cm}\)
%\end{enumerate}